%%%%%%%%%%%%%%%%%%%%%%%%%% ch3
\begin{frame}
  \frametitle{主要内容}
  \tableofcontents[hideallsubsections]
\end{frame}

\section{准备知识}
\begin{frame}
\begin{theorem}
	如果函数$f(x)$在区间$[a,b]$上连续,则积分上限函数
	\[\Phi(x)=\int_a^x f(t)dt\]
	在[a,b]上具有导数,并且它的导数是
	\[\Phi^\prime(x)=\frac{d}{dx}\int_a^xf(t)dt=f(x)\qquad (a\le x\le b) \]
\end{theorem}
\begin{theorem}
	如果函数$f(x)$在区间$[a,b]$上连续,则函数
	\[\Phi(x)=\int_a^x f(t)dt\]	
	就是$f(x)$在$[a,b]$上的一个原函数。
\end{theorem}
\end{frame}

\begin{frame}
\begin{theorem}
	如果函数$F(x)$是连续函数$f(x)$在区间$[a,b]$上的一个原函数,则
	\[\int_a^b f(x)dx=F(b)-F(a)\]	
\end{theorem}
\end{frame}

\begin{frame}
似然比检验的判别式:
\[\lambda(x)=\frac{p(x|H_1)}{p(x|H_0)}\mathop{\gtrless}_{H_0}^{H_1}\eta \]
判决概率:
\[P_F=P(H_1|H_0)=\int_{\eta}^{\infty}p(\lambda|H_0)d\lambda \]
\[P_D=P(H_1|H_1)=\int_{\eta}^{\infty}p(\lambda|H_1)d\lambda \]
\end{frame}

\begin{frame}
\begin{align*}
&P_D =P_D=P(H_1|H_1)=\int_{\eta}^{\infty}p(\lambda|H_1)d\lambda=P_D(\eta) \\
&P_F =P(H_1|H_0)=\int_{\eta}^{\infty}p(\lambda|H_0)d\lambda=P_F(\eta) \\
&\frac{dP_D(\eta)}{d\eta} =-p(\eta|H_1) \\ 
&\frac{dP_F(\eta)}{d\eta} =-p(\eta|H_0) \\ 
&\qquad \text{ by } \Phi^\prime(x)=\frac{d}{dx}\int_a^xf(t)dt=f(x)\qquad (a\le x\le b) \\
&\frac{dP_D(\eta)}{dP_F(\eta)} =\frac{-p(\eta|H_1)}{-p(\eta|H_0)}=\frac{p(\eta|H_1)}{p(\eta|H_0)}\\
\end{align*}
\end{frame}

\begin{frame}
\begin{align*}
P_D(\eta) &=P[(\lambda|H_1)\ge\eta]&\\
&=\int_{\eta}^{\infty}p(\lambda|H_1)d\lambda&\\
&=\int_{R_1}^{\infty}p(x|H_1)dx&\\
&=\int_{R_1}^{\infty}\lambda p(x|H_0)dx &\text{ by }\lambda(x)=\frac{p(x|H_1)}{p(x|H_0)}\mathop{\gtrless}_{H_0}^{H_1}\eta&\\
&=\int_{\eta}^{\infty}\lambda p(\lambda|H_0)d\lambda&
\end{align*}
\[\frac{dP_D(\eta)}{d\eta}=-\eta p(\eta|H_0)\]
\[\frac{dP_D(\eta)}{dP_F(\eta)}=\frac{-p(\eta|H_1)}{-p(\eta|H_0)}=\frac{-\eta p(\eta|H_0)}{-p(\eta|H_0)}=\eta \]  
\end{frame}

\begin{frame}
$H_1$含随机变量$m$的似然比检验的判别式:
\[\lambda(x)=\frac{p(x|m; H_1)}{p(x|H_0)}=\frac{\int_{-\infty}^{\infty}p(x|m,H_1)p(m)dm}{p(x|H_0)}\mathop{\gtrless}_{H_0}^{H_1}\eta \]
\end{frame}

\begin{frame}
$p(m)$未知
\begin{align*}
&p(x|H_0)=(\frac{1}{2\pi\sigma_n^2})^{1/2}\exp(-\frac{x^2}{2\sigma_n^2})\\
&p(x|m;H_1)=(\frac{1}{2\pi\sigma_n^2})^{1/2}\exp(-\frac{(x-m)^2}{2\sigma_n^2})\\
&\lambda(x)=\frac{p(x|m; H_1)}{p(x|H_0)}\mathop{\gtrless}_{H_0}^{H_1}\eta\\
&\exp(\frac{2mx}{2\sigma_n^2}-\frac{m^2}{2\sigma_n^2})\mathop{\gtrless}_{H_0}^{H_1}\eta\\
&mx\mathop{\gtrless}_{H_0}^{H_1}\sigma_n^2\ln\eta+\frac{m^2}{2}
\end{align*}
\end{frame}

\begin{frame}
\begin{align*}
&m_0\le m\le m_1,m_0>0\\
&mx\mathop{\gtrless}_{H_0}^{H_1}\sigma_n^2\ln\eta+\frac{m^2}{2}\\
&l(x)=x\mathop{\gtrless}_{H_0}^{H_1}\frac{\sigma_n^2}{m}\ln\eta+\frac{m}{2}\mathop{=}^{def}\gamma^+\\
&\int_{\gamma^+}^{\infty}(\frac{1}{2\pi\sigma_n^2})^{1/2}\exp(-\frac{l^2}{2\sigma_n^2})dl=\alpha
\end{align*}
\end{frame}

\begin{frame}
\begin{align*}
&m_0\le m\le m_1,m_1<00\\
&mx\mathop{\gtrless}_{H_0}^{H_1}\sigma_n^2\ln\eta+\frac{m^2}{2}\\
&l(x)=x\mathop{\lessgtr}_{H_0}^{H_1}-\frac{\sigma_n^2}{|m|}\ln\eta-\frac{|m|}{2}\mathop{=}^{def}\gamma^-\\
&\int_{-\infty}^{\gamma^-}(\frac{1}{2\pi\sigma_n^2})^{1/2}\exp(-\frac{l^2}{2\sigma_n^2})dl=\alpha
\end{align*}
\end{frame}

\begin{frame}
若$m_0>0$, $m$仅取正值, 则在$P(H_1|H_0)=\alpha$的约束下, $P^{(m)}(H_1|H_1)$是最大的,其一致最大功效检验成立;\\
若$m_1<0$, $m$仅取负值, 则在$P(H_1|H_0)=\alpha$的约束下, $P^{(m)}(H_1|H_1)$也是最大的。
\end{frame}

\begin{frame}
若$m_0<0,m_1>0$, 即$m$取值可能为正或可能为负的情况下, 无论参量信号的统计检测,按$m$仅取正值设计,还是按$m$仅取负值设计,都有可能在某些$m$值下,   $P^{(m)}(H_1|H_1)$不满足最大的要求。\\
例如,按$m$取正设计信号检测系统,当$m$为正时,$P^{(m)}(H_1|H_1)$最大, 但当$m$为负时, $P^{(m)}(H_1|H_1)$可能最小。\\
因此, 这种情况下不能采用奈曼-皮尔逊准则来实际最佳检测系统。
\end{frame}

\begin{frame}
若$m_0<0,m_1>0$, 即$m$取值可能为正或可能为负,奈曼-皮尔逊准则不能保证$P^{(m)}(H_1|H_1)$最大要求。考虑把约束条件$P(H_1|H_0)=\alpha$分成两个$\alpha/2$, 假设$H_1成立$的判决域由两部分组成。判决表示式为
	\[|x|\mathop{\gtrless}_{H_0}^{H_1}\gamma \]

\begin{block}{}
	虽然双边检验比均值$m$假定为正确时的单边检验性能差,但是比均值$m$假定为错误时的单边检验性能要好的多。因此不失为一种好的折中方法。
\end{block}
\end{frame}

\begin{frame}{广义似然比检验}
\begin{align*}
&\text{似然函数}\\
&p(x|m;H_1)=(\frac{1}{2\pi\sigma_n^2})^{1/2}\exp(-\frac{(x-m)^2}{2\sigma_n^2})\\
&\text{对$m$求偏导,令结果等于零,即} \\
&\frac{\partial\ln p(x|m;H_1)}{\partial m}|_{m=\widehat{m}_{ml}}=0\\
&\text{解得单次观测时,$m$的最大似然估计量$\widehat{m}_{ml}=x$, 于是有 }\\
&p(x|\widehat{m}_{ml};H_1)=(\frac{1}{2\pi\sigma_n^2})^{1/2}\exp(-\frac{(x-\widehat{m}_{ml})^2}{2\sigma_n^2})|_{\widehat{m}_{ml}=x}=(\frac{1}{2\pi\sigma_n^2})^{1/2}\\
\end{align*}
\end{frame}
\begin{frame}{广义似然比检验}
\begin{align*}
&p(x|H_0)=(\frac{1}{2\pi\sigma_n^2})^{1/2}\exp(-\frac{x^2}{2\sigma_n^2})
&p(x|\widehat{m}_{ml};H_1)=(\frac{1}{2\pi\sigma_n^2})^{1/2}\\
&\text{代入广义似然比检验中, 有}\\
&\lambda(x)=\frac{p(x|m; H_1)}{p(x|H_0)}\mathop{\gtrless}_{H_0}^{H_1}\eta\\
&\lambda(x)=\frac{(\frac{1}{2\pi\sigma_n^2})^{1/2}}{(\frac{1}{2\pi\sigma_n^2})^{1/2}\exp(-\frac{x^2}{2\sigma_n^2})}\mathop{\gtrless}_{H_0}^{H_1}\eta\\
&\text{化简得判决表示式}
&x^2\mathop{\gtrless}_{H_0}^{H_1}2\sigma_n^2\ln\eta\mathop{=}^{def}\gamma^2 \implies |x|\mathop{\gtrless}_{H_0}^{H_1}\gamma
\end{align*}
这正是前面讨论过的双边检验。只是前面是从奈曼-皮尔逊准则出发推导得到。而这里是从似然比检验的概念导出的,似然函数$p(x|m;H_1)$中的信号参量$m$由其最大似然估计量$\widehat{m}_{ml}$代换,所以是广义似然比检验。
\end{frame}


