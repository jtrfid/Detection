%%%%%%%%%%%%%%%%%%%%%%%%%% ch1
\begin{frame}
  \frametitle{主要内容}
  \tableofcontents[hideallsubsections]
\end{frame}

\section{信号的随机性}
\begin{frame}
\begin{itemize}
	\item \textcolor{blue}{\textbf{信号分类}}
	\begin{itemize}
		\item[-] \textbf{确知信号}: 可以用一个确定的时间函数来表示的信号。\textcolor{red}{$s(t)(0\le t\le T)$}
		\item[-] \textbf{随机未知参量信号}:也表示为时间的函数,但信号中含有一个或一个以上的参量是随机(未知)的。\textcolor{red}{$s(t;\bm{\theta})(0\le t\le T)$}, 其中$\bm{\theta}=(\theta_1,\theta_2,\dots,\theta_M)^T$, 表示信号中含有$M$个随机(未知)参量。
	\end{itemize}
    \item \textcolor{blue}{\textbf{考虑加性噪声$n(t)$下的待处理信号(称作接收信号或观测信号)$x(t)$}}
    \begin{align*}
    \textbf{确知信号: }\qquad &x(t)=s(t)+n(t), &&(0\le t\le T)\\
    \textbf{随机(未知)参量信号: }\qquad &x(t)=s(t;\bm{\theta})+n(t), &&(0\le t\le T)
    \end{align*}
    \item \textcolor{blue}{\textbf{信号的随机性}}\\
    由于噪声$n(t)$是具有随机特性的随机过程, 即使信号是确知信号$s(t)$, 其接收信号$x(t)$也具有随机特性的信号。对于随机(未知)参量信号$s(t;\bm{\theta})$,其接收信号$x(t)$更是一随机信号。
\end{itemize}
\end{frame}

\begin{frame}
\[R=\frac{1}{2}ct_d \]
\end{frame}


