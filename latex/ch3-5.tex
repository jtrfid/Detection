%%%%%%%%%%%%%%%%%%%%%%%%%%ch3-3
\begin{frame}[shrink]
  \frametitle{ch3.信号检测与估计理论的基础知识}
  \framesubtitle{ch3-3. 派生贝叶斯准则(续),信号统计检测的性能}
  \tableofcontents[hideallsubsections]
\end{frame}

\section{极小极大化准则(续)}

\begin{frame}[shrink]{极小极大化准则}
给定$P_{1g}$的条件下, 平均代价$C(P_1,P_{1g})$是先验概率$P_1$的线性函数, 若$P_{1g}\ne P_1$, 平均代价$C(P_1,P_{1g})$大于最小平均代价。\\
为避免产生过分大的代价, 需要猜测一种先验概率$P_{1g}^\ast$, 使得平均代价$C(P_1,P_{1g}^\ast)$不依赖于信源的先验概率$P_1$。
\begin{columns}
	\column{0.7\textwidth}
	\begin{align*}
	&C(P_1,P_{1g})=c_{00}+(c_{10}-c_{00})P_F(P_{1g})+\\
	&P_1[(c_{11}-c_{00})+(c_{01}-c_{11})P_M(P_{1g})-(c_{10}-c_{00})P_F(P_{1g})]\\
	&\frac{\partial C(P_1,P_{1g})}{\partial P_1}\left|_{P_{1g}=P_{1g}^\ast}\right.=0\\
	&\textbf{\textcolor{blue}{极小化极大方程}}\\
	&(c_{11}-c_{00})+(c_{01}-c_{11})P_M(P_{1g}^\ast)-(c_{10}-c_{00})P_F(P_{1g}^\ast)=0
	\end{align*}
	\column{0.3\textwidth}
	\centering
	\includegraphics[scale=0.25]{C-P1}
\end{columns}
~\\
\textbf{\textcolor{blue}{平均代价: }} $C(P_{1g}^\ast)=c_{00}+(c_{10}-c_{00})P_F(P_{1g}^\ast)$
\end{frame}

\begin{frame}[shrink]{极小极大化准则}
\textbf{\textcolor{blue}{极小化极大方程}}
\[(c_{11}-c_{00})+(c_{01}-c_{11})P_M(P_{1g}^\ast)-(c_{10}-c_{00})P_F(P_{1g}^\ast)=0 \]
\textbf{\textcolor{blue}{平均代价: }} \[C(P_{1g}^\ast)=c_{00}+(c_{10}-c_{00})P_F(P_{1g}^\ast) \]
\begin{block}{正确判决不付出代价}
\begin{columns}
	\column{0.3\textwidth}
	$c_{11}=c_{00}=0$
	\column{0.5\textwidth}
	$c_{01}P_M(P_{1g}^\ast)=c_{10}P_F(P_{1g}^\ast)$ 
\end{columns}
\end{block}
\begin{block}{正确判决不付出代价, 错误判决代价因子相同}
\begin{columns}
	\column{0.3\textwidth}
	$c_{11}=c_{00}=0$\\
	$c_{10}=c_{01}=1$
	\column{0.5\textwidth}
	$P_M(P_{1g}^\ast)=P_F(P_{1g}^\ast)$ 
\end{columns}
\end{block}
\end{frame}

\begin{frame}{极小化极大准则的基本步骤}
\begin{enumerate}
\setlength{\itemsep}{.5cm}
\item 计算两个似然函数, 构建似然比$\lambda(x)\mathop{=}\limits^{def}\frac{p(x|H_1)}{p(x|H_0)}$
\item 假设判决门限$\eta$, 构建贝叶斯检测基本表达式
\item 化简成最简形式$l(x)\mathop{\gtrless}\limits_{H_0}^{H_1}\gamma(\eta)$
\item 利用极小化极大准则, 确定最终判决门限$\gamma(\eta)$
\end{enumerate}
\end{frame}

\begin{frame}{贝叶斯准则例题6}
在闭启键控通信系统中,两个假设下的观测信号模型为:
\begin{align*}
H_0: x&=n  \\
H_1: x&=A+n
\end{align*}
其中, 噪声$n$是均值为零,方差为$\sigma_n^2$的高斯噪声,  若两个假设的先验概率未知, 且$c_{00}=c_{11}=0, c_{01}=c_{10}=1$。\\
~\\
采用极小化极大准则,试确定检测门限, 并求最小平均错误概率。
\end{frame}

\begin{frame}[shrink]{贝叶斯准则例题6: 解}
解: 观测信号模型为:
\begin{align*}
H_0: x&=n  \\
H_1: x&=A+n
\end{align*}
\textbf{步骤1: 计算两个似然函数, 构建似然比}\\
由于$n$是高斯分布随机变量, 因此在$H_0$假设下, 检测统计量$x$服从高斯分布,且均值为0, 方差为$\sigma_n^2$; 在$H_1$假设下, 检测统计量$x$服从均值为$A$, 方差为$\sigma_n^2$的高斯分布。
\begin{align*}
&p(x|H_0)=\left(\frac{1}{2\pi\sigma_n^2}\right)^{1/2}\exp\left(-\frac{x^2}{2\sigma_n^2}\right) \qquad p(x|H_1)=\left(\frac{1}{2\pi\sigma_n^2}\right)^{1/2}\exp\left(-\frac{(x-A)^2}{2\sigma_n^2}\right)\\
&\lambda(x)=\frac{p(x|H_1)}{p(x|H_0)}=\exp\left(\frac{(x^2-(x-A)^2)}{2\sigma_n^2}\right)=\exp\left(\frac{A}{\sigma_n^2}x-\frac{A^2}{2\sigma_n^2}\right)
\end{align*} 
\end{frame}

\begin{frame}[shrink]{贝叶斯准则例题6: 解(续1)}
\textbf{步骤2: 假设判决门限$\eta$, 构建贝叶斯检测基本表达式}
\begin{align*}
&\lambda(x)\mathop{=}^{def}\frac{p(x|H_1)}{p(x|H_0)}\mathop{\gtrless}_{H_0}^{H_1}\eta\\
&\lambda(x)=\exp\left(\frac{A}{\sigma_n^2}x-\frac{A^2}{2\sigma_n^2}\right)
\end{align*} 
\textbf{步骤3: 化简成最简形式}
\begin{align*}
&x\mathop{\gtrless}_{H_0}^{H_1}\frac{\sigma_n^2\ln\eta}{A}+\frac{A}{2}\mathop{=}^{def}\gamma
\end{align*} 
\end{frame}

\begin{frame}[shrink]{贝叶斯准则例题6: 解(续2)}
\textbf{步骤4: 利用极小化极大准则, 确定最终判决门限$\gamma$}
\includegraphics[scale=0.3]{R0R1}
\begin{align*}
P_F\mathop{=}^{def}P(H_1|H_0)&=\int_{\gamma}^{\infty}p(x|H_0)dx\implies Q(x)=\int_{x}^{\infty}\left(\frac{1}{2\pi}\right)^{1/2}\exp\left(-\frac{u^2}{2}\right)du\\
&=\int_{\gamma}^{\infty}\left(\frac{1}{2\pi\sigma_n^2}\right)^{1/2}\exp\left(-\frac{x^2}{2\sigma_n^2}\right)dx\qquad \text{by } x=\sigma_nu\\
&=\int_{\frac{\gamma}{\sigma_n}}^{\infty}\left(\frac{1}{2\pi}\right)^{1/2}\exp\left(-\frac{u^2}{2}\right)du\\
&=Q\left(\frac{\gamma}{\sigma_n}\right)
\end{align*}
\end{frame}

\begin{frame}[shrink]{贝叶斯准则例题6: 解(续3)}
\includegraphics[scale=0.3]{R0R1}
\begin{align*}
P_M\mathop{=}^{def}P(H_0|H_1)&=1-\int_{\gamma}^{\infty}p(x|H_1)dx\implies Q(x)=\int_{x}^{\infty}\left(\frac{1}{2\pi}\right)^{1/2}\exp\left(-\frac{u^2}{2}\right)du\\
&=1-\int_{\gamma}^{\infty}\left(\frac{1}{2\pi\sigma_n^2}\right)^{1/2}\exp\left(-\frac{(x-A)^2}{2\sigma_n^2}\right)dx\qquad \text{by } x=\sigma_nu+A\\
&=1-\int_{\frac{\gamma-A}{\sigma_n}}^{\infty}\left(\frac{1}{2\pi}\right)^{1/2}\exp\left(-\frac{u^2}{2}\right)du\\
&=1-Q\left(\frac{\gamma-A}{\sigma_n}\right)
\end{align*}
\end{frame}

\begin{frame}[shrink]{贝叶斯准则例题6: 解(续4)}
\begin{block}{正确判决不付出代价, 错误判决代价因子相同时的极小化极大方程}
\begin{columns}
\column{0.3\textwidth}
$c_{11}=c_{00}=0$\\
$c_{10}=c_{01}=1$
\column{0.5\textwidth}
$P_M(P_{1g}^\ast)=P_F(P_{1g}^\ast)$ 
\end{columns}
\end{block}
\begin{columns}[T]
\column{0.5\textwidth}
\small
\begin{align*}
P_F\mathop{=}^{def}P(H_1|H_0)&=Q\left(\frac{\gamma}{\sigma_n}\right)\\
P_M\mathop{=}^{def}P(H_0|H_1)&=1-Q\left(\frac{\gamma-A}{\sigma_n}\right)\\
&=Q\left(-\frac{\gamma-A}{\sigma_n}\right)
\end{align*}
根据上述极小化极大方程,有
\begin{align*}
Q\left(\frac{\gamma}{\sigma_n}\right)=Q\left(-\frac{\gamma-A}{\sigma_n}\right)\implies \gamma=\frac{A}{2}
\end{align*}
\column{0.5\textwidth}
\small
\begin{align*}
&Q(x)=\int_{x}^{\infty}\left(\frac{1}{2\pi}\right)^{1/2}\exp\left(-\frac{u^2}{2}\right)du\\
&Q\left(\frac{d}{2}\right) = 1-Q\left(-\frac{d}{2}\right)
\end{align*}
\centering
\includegraphics[scale=0.25]{Qdd}
\end{columns}
\end{frame}

\begin{frame}[shrink]{贝叶斯准则例题6: 解(续5)}
\small
\textbf{本例, 按照极小化极大准则,平均错误概率为:}
\begin{align*}
P_e&=P(H_1)P(H_0|H_1)+P(H_0)P(H_1|H_0)\\
&=P(H_1)P_M+P(H_0)P_F\\
&=[P(H_1)+P(H_0)]P_F &&\text{by 本例的极小化极大方程}P_M(P_{1g}^\ast)=P_F(P_{1g}^\ast)\\
&=P_F=P(H_1|H_0) &&\text{by }P(H_1)+P(H_0)=1, P_F\mathop{=}^{def}P(H_1|H_0)\\
&=Q(\frac{\gamma}{\sigma})=Q\left(\frac{A}{2\sigma}\right) && \text{by }\gamma=\frac{A}{2}\\
&=Q(\frac{d}{2}) &&\text{by 功率信噪比}d^2=\frac{A^2}{\sigma^2}
\end{align*}
\textbf{例题5, 按照按照平均错误概率准则,平均错误概率同上。}\\
因此, \textbf{\textcolor{blue}{先验等概条件下的最小平均错误准则等价于正确判决为0, 错误判决代价为1时的极小化极大准则。}}
\end{frame}

\section{奈曼---皮尔逊准则}

\begin{frame}[shrink]{奈曼---皮尔逊准则(Neyman-Pearson criterion)}
\begin{columns}
\column{0.72\textwidth}
\begin{itemize}
%\setlength{\itemsep}{.5cm}
\item \textbf{\textcolor{blue}{应用范围}}\\
假设的先验概率未知, 判决代价未知(雷达信号检测)
\item \textbf{\textcolor{blue}{目标}}\\
错误判决概率尽可能小, 正确判决概率尽可能大
\item \textbf{\textcolor{blue}{实际情况}}\\
$P(H_1|H_0)$减小时, $P(H_1|H_1)$也相应减小;\\
增加$P(H_1|H_1)$, $P(H_1|H_0)$也随之增加。
\item \textbf{\textcolor{blue}{奈曼皮尔逊检测}}\\
在虚警概率$P_F\mathop{=}\limits^{def}P(H_1|H_0)=\alpha$约束条件下, 使正确判决概率(检测概率)$P_D\mathop{=}\limits^{def}P(H_1|H_1)$最大的准则。
\end{itemize}
\column{0.28\textwidth}
\leftline{\includegraphics[scale=0.17]{R0R1}}
\end{columns}
\end{frame}

\begin{frame}{奈曼---皮尔逊准则的存在性}
\begin{columns}
\column{0.5\textwidth}
\begin{enumerate}
\item 图中,三个判决域$(R_{0i},R_{1i})$均满足错误判决概率$P_i(H_1|H_0)=\alpha(i=0,1,2)$。
\item 原则上判决域$R_0$和$R_1$有无限多种划分方法,均可以保证错误判决概率$P(H_1|H_0)=\alpha$, 但是正确判决概率$P(H_1|H_1)$一般是不一样的。
\item 至少有一种判决域划分能使$P(H_1|H_0)=\alpha$,又能使$P(H_1|H_1)$到达最大。
\end{enumerate}
\column{0.4\textwidth}
\includegraphics[scale=0.45]{Neyman-Pearson}
\end{columns}
\end{frame}

\begin{frame}[shrink]{奈曼---皮尔逊准则的推导}
在$P(H_1|H_0)=\alpha$约束条件下, 使正确判决概率$P(H_1|H_1)$最大的准则\\
\qquad\qquad\qquad $\Updownarrow$\quad(由于$P(H_0|H_1)+P(H_1|H_1)=1$)\\
在$P(H_1|H_0)=\alpha$约束条件下, 使错误判决概率$P(H_0|H_1)$最小的准则
\rule{\textwidth}{1pt} %水平线
利用拉格朗日乘子$\mu(\mu\ge 0)$, 构建目标函数
\[ J=P(H_0|H_1)+\mu\left[P(H_1|H_0)-\alpha\right]\]
若$P(H_1|H_0)=\alpha, J$达到最小时, $P(H_0|H_1)$也达到最小。
\begin{columns}
\scriptsize
\column{0.5\textwidth}
漏警概率$P(H_0|H_1)+$检测概率$P(H_1|H_1)=1$,\\ 虚警概率$P(H_1|H_0)=\alpha$\\
当$J$最小$\implies$漏警概率$(P(H_0|H_1)$最小\\
$\implies$检测概率$P(H_1|H_1)$最大。
\column{0.5 \textwidth}
\leftline{\includegraphics[scale=0.25]{R0R1}}
\end{columns}
\end{frame}

\begin{frame}[shrink]{奈曼---皮尔逊准则的推导(续)}
\begin{columns}
\column{0.4\textwidth}
\begin{align*}
J&=P(H_0|H_1)+\mu[P(H_1|H_0)-\alpha]\\
&=\int_{R_0}p(x|H_1)dx+\mu\left[\int_{R_1}p(x|H_0)dx-\alpha\right]\\
&=\int_{R_0}p(x|H_1)dx+\mu\left[1-\int_{R_0}p(x|H_0)dx-\alpha\right]\\
&=\mu(1-\alpha)+\int_{R_0}\left[p(x|H_1)-\mu p(x|H_0)\right]dx \\
\end{align*}
\column{0.4\textwidth}
\leftline{\includegraphics[scale=0.23]{R0R1}}
\end{columns}
\textbf{\textcolor{blue}{把使被积函数取负值的观测值$x$值划分给$R_0$区域, 而把其余的观测值$x$值划分给$R_1$, 即可保证平均代价最小, 从而使$J$值最小。}}
\begin{align*}
&p(x|H_1)< \mu p(x|H_0)&\textbf{判决$H_0$假设成立}\\
&p(x|H_1)\ge \mu p(x|H_0)&\textbf{判决$H_1$假设成立}
\end{align*}
\end{frame}

\begin{frame}[shrink]{奈曼---皮尔逊准则}
\begin{block}{奈曼---皮尔逊准则}
\[ \lambda(x)\mathop{=}\limits^{def}\frac{p(x|H_1)}{p(x|H_0)}\mathop{\gtrless}_{H_0}^{H_1}\mu \]
其中, 判决门限有下式确定
\begin{align*}
P(H_1|H_0)=\int_{R_1}p(x|H_0)dx=\int_{\mu}^{\infty}p(\lambda|H_0)d\lambda=\alpha\
\end{align*}
求出的$\mu$必满足$\mu\ge 0$
\end{block}
\begin{block}{贝叶斯判决准则}
\[ \lambda(x)=\frac{p(x|H_1)}{p(x|H_0)}\mathop{\gtrless}_{H_0}^{H_1}\frac{P(H_0)(c_{10}-c_{00})}{P(H_1)(c_{01}-c_{11})}\mathop{=}^{def}\eta \]
\end{block}
贝叶斯准则的特例, 当$P(H_1)(c_{01}-c_{11})=1, P(H_0)(c_{10}-c_{00})=\mu$时, 就成为奈曼---皮尔逊准则。
\end{frame}

\begin{frame}[shrink]{奈曼---皮尔逊准则的求解步骤}
\begin{enumerate}
\setlength{\itemsep}{.5cm}
\item 计算两个似然函数, 构建似然比$\lambda(x)\mathop{=}\limits^{def}\frac{p(x|H_1)}{p(x|H_0)}\mathop{\gtrless}\limits_{H_0}^{H_1}\mu$
\item 假设判决门限$\mu$, 构建贝叶斯检测基本表达式
\item 化简
\item 根据统计量计算$p(l|H_0)$和$p(l|H_1)$
\item 在$P(H_1|H_0)=\int_{R_1}p(l|H_0)dl=\alpha$约束下, 计算判决门限
\end{enumerate}
\end{frame}

\begin{frame}{贝叶斯准则例题7}
在闭启键控通信系统中,两个假设下的观测信号模型为:
\begin{align*}
H_0: x&=n  \\
H_1: x&=1+n
\end{align*}
其中, 噪声$n$是均值为零,方差为1的高斯噪声。\\
~\\
试构造在$P(H_1|H_0)=0.1$条件下的奈曼---皮尔逊接收机
\end{frame}

\begin{frame}[shrink]{贝叶斯准则例题7: 解}
解: 观测信号模型为:
\begin{align*}
H_0: x&=n  \\
H_1: x&=1+n
\end{align*}
\textbf{步骤1: 计算两个似然函数, 构建似然比}\\
由于$n$是高斯分布随机变量, 因此在$H_0$假设下, 检测统计量$x$服从高斯分布,且均值为0, 方差为$\sigma_n^2$; 在$H_1$假设下, 检测统计量$x$服从均值为$1$, 方差为$\sigma_n^2$的高斯分布。
\begin{align*}
&p(x|H_0)=\left(\frac{1}{2\pi\sigma_n^2}\right)^{1/2}\exp\left(-\frac{x^2}{2\sigma_n^2}\right) \qquad p(x|H_1)=\left(\frac{1}{2\pi\sigma_n^2}\right)^{1/2}\exp\left(-\frac{(x-1)^2}{2\sigma_n^2}\right)\\
&\lambda(x)=\frac{p(x|H_1)}{p(x|H_0)}=\exp\left(\frac{(x^2-(x-1)^2)}{2\sigma_n^2}\right)=\exp\left(\frac{1}{\sigma_n^2}x-\frac{1}{2\sigma_n^2}\right)
\end{align*} 
\end{frame}

\begin{frame}[shrink]{贝叶斯准则例题7: 解(续1)}
\textbf{步骤2: 假设判决门限$\mu$, 构建贝叶斯检测基本表达式}
\begin{align*}
&\lambda(x)\mathop{=}^{def}\frac{p(x|H_1)}{p(x|H_0)}\mathop{\gtrless}_{H_0}^{H_1}\mu\\
&\lambda(x)=\exp\left(\frac{1}{\sigma_n^2}x-\frac{1}{2\sigma_n^2}\right)
\end{align*} 
\textbf{步骤3: 化简成最简形式}
\begin{align*}
&x\mathop{\gtrless}_{H_0}^{H_1}\sigma_n^2\ln\mu+\frac{1}{2}\mathop{=}^{def}\gamma\\
&\text{by }\sigma_n=1\\
&x\mathop{\gtrless}_{H_0}^{H_1}\ln\mu+\frac{1}{2}\mathop{=}^{def}\gamma\\
\end{align*} 
\end{frame}

\begin{frame}[shrink]{贝叶斯准则例题7: 解(续2)}
\textbf{步骤4: 利用奈曼---皮尔逊准则, 确定最终判决门限$\gamma$}
\includegraphics[scale=0.3]{R0R1}
\begin{align*}
P_F\mathop{=}^{def}P(H_1|H_0)&=\int_{\gamma}^{\infty}p(x|H_0)dx\implies Q(x)=\int_{x}^{\infty}\left(\frac{1}{2\pi}\right)^{1/2}\exp\left(-\frac{u^2}{2}\right)du\\
&=\int_{\gamma}^{\infty}\left(\frac{1}{2\pi\sigma_n^2}\right)^{1/2}\exp\left(-\frac{x^2}{2\sigma_n^2}\right)dx\qquad \text{by } x=\sigma_nu\\
&=\int_{\frac{\gamma}{\sigma_n}}^{\infty}\left(\frac{1}{2\pi}\right)^{1/2}\exp\left(-\frac{u^2}{2}\right)du\\
&=Q\left(\frac{\gamma}{\sigma_n}\right)=Q(\gamma)\qquad \text{by }\sigma_n=1
\end{align*}
\end{frame}

\begin{frame}[shrink]{贝叶斯准则例题7: 解(续3)}
\begin{columns}
\column{0.55\textwidth}
\begin{align*}
&x\mathop{\gtrless}_{H_0}^{H_1}\ln\mu+\frac{1}{2}\mathop{=}^{def}\gamma\\
&P(H_1|H_0)=Q(\gamma)
\end{align*} 
在$P(H_1|H_0)=0.1$条件下,确定判决门限\\
由$Q(\gamma)=0.1$, 解得$\gamma=1.29$,\\
由$\ln\mu+\frac{1}{2}=\gamma$, 解得$\mu=2.2$
\column{0.4\textwidth}
\leftline{\includegraphics[scale=0.3]{ex3-7}}
\end{columns}
\begin{align*}
P_D\mathop{=}^{def}P(H_1|H_1)&=\int_{\gamma}^{\infty}p(x|H_1)dx\implies Q(x)=\int_{x}^{\infty}\left(\frac{1}{2\pi}\right)^{1/2}\exp\left(-\frac{u^2}{2}\right)du\\
&=\int_{\gamma}^{\infty}\left(\frac{1}{2\pi\sigma_n^2}\right)^{1/2}\exp\left(-\frac{(x-1)^2}{2\sigma_n^2}\right)dx\qquad \text{by } x=\sigma_nu+1\\
&=\int_{\frac{\gamma-1}{\sigma_n}}^{\infty}\left(\frac{1}{2\pi}\right)^{1/2}\exp\left(-\frac{u^2}{2}\right)du\\
&=Q\left(\frac{\gamma-1}{\sigma_n}\right)=Q(\gamma-1)=Q(0.29)=0.386
\end{align*}
\end{frame}

\begin{frame}[shrink]{贝叶斯准则以及派生贝叶斯准则(1)}
\includegraphics[scale=0.5]{bys}
\end{frame}

\begin{frame}[shrink]{贝叶斯准则以及派生贝叶斯准则(2)}
\includegraphics[scale=0.5]{bys2}
\end{frame}

\begin{frame}[shrink]{贝叶斯准则以及派生贝叶斯准则求解步骤}
分析某种检测方法得性能时,需要根据化简后得最简判决表示式进行。\\
计算步骤:\\
\begin{enumerate}
%\setlength{\itemsep}{.5cm}
\item 推导某种检测方法下获得的最简判决表达式$l(x)\mathop{\gtrless}\limits_{H_0}^{H_1}\gamma$
\item 根据最简表示式, 计算各种假设下, 统计量的概率密度函数
\[p(l|H_0)\qquad p(l|H_1) \]
\item 计算判决概率\\
\[P(H_0|H_1)\qquad P(H_1|H_0) \]
\end{enumerate}
\end{frame}

\begin{frame}{信号统计检测的性能}
\begin{block}{基本要求}
	\begin{itemize} \setstretch{2.5}
		\item 理解判决概率的不同计算方法
		\item 理解似然比检测的接收机工作特性
		\item 利用接收机工作特性求解不同检测准则的解	
	\end{itemize}
\end{block}
\end{frame}

\section{信号统计检测的性能}

\begin{frame}[shrink]{信号统计检测的性能}
\begin{itemize}
	\item \textbf{\textcolor{blue}{判决概率计算}}
		\begin{align*}
		&\lambda(\bm{x})=\frac{p(\bm{x}|H_1)}{p(\bm{x}|H_0)}\mathop{\gtrless}_{H_0}^{H_1}\eta && l(\bm{x}) \mathop{\gtrless}_{H_0}^{H_1}\gamma\\
		&P(H_1|H_0)=\int_{\eta}^{\infty}p(\lambda|H_0)d\lambda && P(H_1|H_0)=\int_{\gamma}^{\infty}p(l|H_0)dl\\
		&P(H_1|H_1)=\int_{\eta}^{\infty}p(\lambda|H_1)d\lambda && P(H_1|H_1)=\int_{\gamma}^{\infty}p(l|H_1)dl
		\end{align*}
	\item \textbf{\textcolor{blue}{似然比检测的接收机工作特性}}\\
根据$P_D=P(H_1|H_1)$和$P_F=P(H_1|H_0)$分析似然比检测的接收机工作特性
\end{itemize}
\rightline{\includegraphics[scale=0.25]{R0R1}}
\end{frame}

\begin{frame}[shrink]{信号统计检测的性能}
\begin{itemize}
	\item \textbf{\textcolor{blue}{判决概率计算}}
	\begin{align*}
	&\lambda(\bm{x})=\frac{p(\bm{x}|H_1)}{p(\bm{x}|H_0)}\mathop{\gtrless}_{H_0}^{H_1}\eta && l(\bm{x}) \mathop{\gtrless}_{H_0}^{H_1}\gamma\\
	&P(H_1|H_0)=\int_{\eta}^{\infty}p(\lambda|H_0)d\lambda && P(H_1|H_0)=\int_{\gamma}^{\infty}p(l|H_0)dl\\
	&P(H_1|H_1)=\int_{\eta}^{\infty}p(\lambda|H_1)d\lambda && P(H_1|H_1)=\int_{\gamma}^{\infty}p(l|H_1)dl
	\end{align*}
	\item \textbf{\textcolor{blue}{似然比检测的接收机工作特性}}\\
	根据$P_D=P(H_1|H_1)$和$P_F=P(H_1|H_0)$分析似然比检测的接收机工作特性
\end{itemize}
\rightline{\includegraphics[scale=0.25]{R0R1}}
\end{frame}

\begin{frame}[shrink]{例如, 雷达信号检测}
\begin{columns}
\column{0.21\textwidth}
$H_0:x_k=n_k$\\
$H_1:x_k=A+n_k$
\column{0.79\textwidth}
\[
\frac{1}{N}\sum\limits_{k=1}^{N}x_k\mathop{\gtrless}\limits_{H_0}^{H_1}\frac{\sigma_n^2\ln\eta}{NA}+\frac{A}{2}\mathop{=}\limits^{def}\gamma \qquad \textbf{统计量: }l(\bm{x})\mathop{=}\limits^{def}\frac{1}{N}\sum\limits_{k=1}^{N}x_k
\]
\end{columns}
\textbf{假设$H_0$条件下, 统计量$l(x)$为高斯分布, $(l|H_0)\sim\mathcal{N}(0,\frac{\sigma_n^2}{N})$}
\begin{align*}
p(l|H_0)=\left(\frac{N}{2\pi\sigma_n^2}\right)^{1/2}\exp\left(-\frac{Nl^2}{2\sigma_n^2}\right)
\end{align*}
\textbf{假设$H_1$条件下, 统计量$l(x)$为高斯分布, $(l|H_1)\sim\mathcal{N}(A,\frac{\sigma_n^2}{N})$}
\begin{align*}
p(l|H_1)=\left(\frac{N}{2\pi\sigma_n^2}\right)^{1/2}\exp\left(-\frac{N(l-A)^2}{2\sigma_n^2}\right)
\end{align*}
\end{frame}

\begin{frame}[shrink]{虚警概率$P_F=P(H_1|H_0)$}
\begin{align*}
P(H_1|H_0)&=\int_{\gamma}^{\infty}p(l|H_0)dl\implies Q(x)=\int_{x}^{\infty}\left(\frac{1}{2\pi}\right)^{1/2}\exp\left(-\frac{u^2}{2}\right)du\\
&=\int_{\gamma}^{\infty}\left(\frac{N}{2\pi\sigma_n^2}\right)^{1/2}\exp\left(-\frac{Nl^2}{2\sigma_n^2}\right)dl\qquad \text{by } l=\frac{\sigma_nu}{\sqrt{N}}\\
&=\int_{\frac{\sqrt{N}\gamma}{\sigma_n}}^{\infty}\left(\frac{1}{2\pi}\right)^{1/2}\exp\left(-\frac{u^2}{2}\right)du\\
&=Q\left(\frac{\sqrt{N}\gamma}{\sigma_n}\right) &&\text{by } \gamma=\frac{\sigma_n^2\ln\eta}{NA}+\frac{A}{2}\\
&=Q\left(\frac{\sqrt{N}\left(\frac{\sigma_n^2\ln\eta}{NA}+\frac{A}{2}\right)}{\sigma_n}\right)\\
&=Q\left(\frac{\sigma_n\ln\eta}{\sqrt{N}A}+\frac{\sqrt{N}A}{2\sigma_n}\right) &&\text{by }d^2=\frac{NA^2}{\sigma_n^2}\\
&=Q\left(\frac{\ln\eta}{d}+\frac{d}{2}\right)
\end{align*}
\end{frame}

\begin{frame}[shrink]{检测概率$P_D=P(H_1|H_1)$}
\begin{align*}
P(H_1|H_1)&=\int_{\gamma}^{\infty}p(l|H_1)dl\implies Q(x)=\int_{x}^{\infty}\left(\frac{1}{2\pi}\right)^{1/2}\exp\left(-\frac{u^2}{2}\right)du\\
&=\int_{\gamma}^{\infty}\left(\frac{N}{2\pi\sigma_n^2}\right)^{1/2}\exp\left(-\frac{N(l-A)^2}{2\sigma_n^2}\right)dl\qquad \text{by } l=\frac{\sigma_nu}{\sqrt{N}}+A\\
&=\int_{\frac{\sqrt{N}(\gamma-A)}{\sigma_n}}^{\infty}\left(\frac{1}{2\pi}\right)^{1/2}\exp\left(-\frac{u^2}{2}\right)du\\
&=Q\left(\frac{\sqrt{N}(\gamma-A}{\sigma_n}\right) &&\text{by } \gamma=\frac{\sigma_n^2\ln\eta}{NA}+\frac{A}{2}\\
&=Q\left(\frac{\sqrt{N}\left(\frac{\sigma_n^2\ln\eta}{NA}-\frac{A}{2}\right)}{\sigma_n}\right)\\
&=Q\left(\frac{\sigma_n\ln\eta}{\sqrt{N}A}-\frac{\sqrt{N}A}{2\sigma_n}\right)&& \text{by }d^2=\frac{NA^2}{\sigma_n^2}\\
&=Q\left(\frac{\ln\eta}{d}-\frac{d}{2}\right)
\end{align*}
\end{frame}

\begin{frame}[shrink]{判决域与判决概率}
\begin{columns}
\column{0.6\textwidth}
$N$次独立采样, 样本为$x_k(k=1,2,\dots,N)$
\begin{align*}
&&n_k\sim\mathcal{N}(0,\sigma_n^2)\\ 
H_0 &:x_k=n_k   &(l|H_0)\sim\mathcal{N}(0,\frac{\sigma_n^2}{N})\\
H_1 &:x_k=A+n_k &(l|H_1)\sim\mathcal{N}(A,\frac{\sigma_n^2}{N})
\end{align*}
检验统计量$l(\bm{x})$, 归一化后, $(l|H_j)\sim\mathcal{N}(0,1)$
\column{0.4\textwidth}
\includegraphics[scale=0.28]{R0R1}\\
\end{columns}
\[
\textbf{判决表达式: }\qquad l(\bm{x})\mathop{=}^{def}\frac{1}{N}\sum\limits_{k=1}^Nx_k\mathop{\gtrless}_{H_0}^{H_1}\frac{\sigma^2\ln\eta}{NA}+\frac{A}{2}\mathop{=}^{def}\gamma
\]

\textbf{判决概率:} (式中, 信噪比$d^2=\frac{NA^2}{\sigma_n^2}$)
\begin{align*}
\textbf{虚警概率: }P_F\mathop{=}^{def}P(H_1|H_0)&=Q\left(\frac{\ln\eta}{d}+\frac{d}{2}\right)\\
\textbf{检测概率: }P_D\mathop{=}^{def}P(H_1|H_1)&=Q\left(\frac{\ln\eta}{d}-\frac{d}{2}\right)
\end{align*}
\end{frame}

\begin{frame}[shrink]{接收机工作特性}
\begin{columns}[T]
	\column{0.6\textwidth}
	\begin{itemize}
		\small
		\item 错误判别概率(虚警概率):
		\[P_F\mathop{=}^{def}P(H_1|H_0)=Q\left(\frac{\ln\eta}{d}+\frac{d}{2}\right)\]
		\item 正确判别概率(检测概率):
		\[P_D\mathop{=}^{def}P(H_1|H_1)=Q\left(\frac{\ln\eta}{d}-\frac{d}{2}\right)\]
		\item 不同的信噪比$d$, 有不同的$P_D\sim P_F$曲线
		\item 似然比函数$\lambda(x)$超过无穷大门限$\eta=+\infty$是不可能事件, $(P_D,P_F)=(0,0)$
		\item $\lambda(x)\ge 0$, $\eta=0$是必然事件, $(P_D,P_F)=(1,1)$
		\item 当$\lambda(x)$是连续随机变量时, $\eta\uparrow\implies (P_D,P_F)\downarrow$
	\end{itemize}
	\column{0.4\textwidth}
	\includegraphics[scale=0.28]{PDPF}\\
	\includegraphics[scale=0.2]{R0R1}
\end{columns}
\end{frame}

\begin{frame}[shrink]{接收机工作特性的共同特点}
\begin{columns}[T]
	\column{0.6\textwidth}
	\begin{itemize}
		\small
		\item 上凸曲线
		\item 曲线位于$P_D=P_F$之上
		\item 随着门限$\eta$的增加, 两种判决概率$P_D$和$P_F$之都会减小
		\item $P_D$和$P_F$同时增加,或同时减小
		\item 给定$P_D(P_F)$, 随着信噪比$d$的增加, $P_F$减小($P_D$增加)
		\item 工作特性某点上的斜率等于该点$P_D$和$P_F$所要求的检测门限值
		\item 利用接收机工作特性,可进行各种判决准则的分析和计算
	\end{itemize}
	\column{0.4\textwidth}
	\includegraphics[scale=0.28]{PDPF}\\
	\includegraphics[scale=0.2]{R0R1}
\end{columns}
\end{frame}

\begin{frame}[shrink]{检测概率$P_D$与信噪比$d$的关系}
信噪比$d$是接收机的主要指标之一,因此常把接收机工作特性改成$P_D\sim d$曲线, 而以$P_F$作为参变量。
\begin{columns}
	\column{0.5\textwidth}
	\begin{align*}
	P_F&=P(H_1|H_0)=Q(\ln\eta/d+d/2)\\ 
	&\ln\eta/d=Q^{-1}[P(H_1|H_0)]-d/2\\
	P_D&=P(H_1|H_1)=Q\left(\ln\eta/d-d/2\right)\\
	&=Q[Q^{-1}[P(H_1|H_0)]-d/2-d/2]\\
	&=Q[Q^{-1}[P(H_1|H_0)]-d]
	\end{align*}
	\column{0.5\textwidth}
	\includegraphics[scale=0.5]{PD-d}
\end{columns}

\medskip

\colorbox{yellow}{$Q(x)$是递减函数, 当给定$P_F$时, $P_D$随功率信噪比$(d^2=NA^2/\sigma_n^2)$单调增加。} 
\end{frame}

\begin{frame}[shrink]{工作特性某点上的斜率等于该点$P_D$和$P_F$所要求的检测门限值$\eta$}
\begin{columns}
	\column{0.5\textwidth}
	\begin{align*}
	&P_D\mathop{=}^{def}P(H_1|H_1)=\int_{\eta}^{\infty}p(\lambda|H_1)d\lambda\mathop{=}^{def}P_D(\eta) \\
	&P_F\mathop{=}^{def}P(H_1|H_0)=\int_{\eta}^{\infty}p(\lambda|H_0)d\lambda\mathop
	{=}^{def}P_F(\eta) \\
	&\frac{dP_D(\eta)}{d\eta} =-p(\eta|H_1) \\ 
	&\frac{dP_F(\eta)}{d\eta} =-p(\eta|H_0) \\ 
	&\qquad \text{ by } \Phi^\prime(x)=\frac{d}{dx}\int_a^xf(t)dt=f(x)\qquad (a\le x\le b) \\
	&\frac{dP_D(\eta)}{dP_F(\eta)} =\frac{-p(\eta|H_1)}{-p(\eta|H_0)}=\frac{p(\eta|H_1)}{p(\eta|H_0)}
	\end{align*}
	\column{0.5\textwidth}
	\includegraphics[scale=0.48]{PD-PF-e}
\end{columns}
\end{frame}

\begin{frame}[shrink]{工作特性某点上的斜率等于该点$P_D$和$P_F$所要求的检测门限值$\eta$}
\begin{columns}[T]
	\column{0.5\textwidth}
	\small
	\begin{align*}
	P_D(\eta) &=P[(\lambda|H_1)\ge\eta]&\\
	&=\int_{\eta}^{\infty}p(\lambda|H_1)d\lambda=\int_{R_1}^{\infty}p(x|H_1)dx&\\
	&=\int_{R_1}^{\infty}\lambda p(x|H_0)dx \qquad \text{ by }\lambda(x)=\frac{p(x|H_1)}{p(x|H_0)}\mathop{\gtrless}_{H_0}^{H_1}\eta&\\
	&=\int_{\eta}^{\infty}\lambda p(\lambda|H_0)d\lambda&\\
	&\text{ by } \Phi^\prime(x)=\frac{d}{dx}\int_a^xf(t)dt=f(x)\qquad (a\le x\le b) \\
	\frac{dP_D(\eta)}{d\eta} &=-\eta p(\eta|H_0)\\
	\frac{dP_D(\eta)}{dP_F(\eta)} &=\frac{-p(\eta|H_1)}{-p(\eta|H_0)}=\frac{-\eta p(\eta|H_0)}{-p(\eta|H_0)}=\eta
	\end{align*}
	\column{0.5\textwidth}
	\includegraphics[scale=0.47]{PD-PF-e}
\end{columns}\end{frame}

\section{利用接收机工作特性,各种判决准则的分析和计算}

\begin{frame}[shrink]{贝叶斯准则和最小错误概率准则}
\begin{columns}[T]
	\column{0.5\textwidth}
		\begin{itemize}
		\item 根据先验概率和代价因子, 求得判决门限$\eta$
		\item 以$\eta$为斜率, 可找到一条直线, 与在给定信噪比$d$下的$P_D-P_F$曲线相切; 如, $d=d_1$, 切点$u$
		\item 切点对应的$P_D$和$P_F$值,就是在给定信噪比下的两种判决概率。
	\end{itemize}
	\column{0.5\textwidth}
	\includegraphics[scale=0.5]{PD-PF-e}
\end{columns}
\end{frame}

\begin{frame}[shrink]{极小化极大准则}
\textbf{\textcolor{blue}{满足极小化极大方程}}
\begin{align*}
&(c_{11}-c_{00})+(c_{01}-c_{11})P_M(P_{1g}^\ast)-(c_{10}-c_{00})P_F(P_{1g}^\ast)=0\\
&(c_{11}-c_{00})+(c_{01}-c_{11})\left(1-P_D(P_{1g}^\ast)\right)-(c_{10}-c_{00})P_F(P_{1g}^\ast)=0\\
&(c_{01}-c_{11})P_D(P_{1g}^\ast)+(c_{10}-c_{00})P_F(P_{1g}^\ast)-c_{01}+c_{00}=0
\end{align*}
\begin{columns}[T]
	\column{0.5\textwidth}
	\begin{align*}
	&P_D \mathop{=}^{def}P_F(P_1)=P(H_1|H_1)\\
	&P_F \mathop{=}^{def}P_F(P_1)=P(H_1|H_0)\\
	&P_M \mathop{=}^{def}P_M(P_1)=P(H_0|H_1)=1-P_D\\
	&P_M(P_{1g}^\ast)=1-P_D(P_{1g}^\ast)
	\end{align*}
	\[\bm{P_M(P_{1g}^\ast)=1-P_D(P_{1g}^\ast)} \]
	\column{0.5\textwidth}
	\includegraphics[scale=0.25]{R0R1}
\end{columns}
\end{frame}

\begin{frame}[shrink]{极小化极大准则}
\begin{columns}[T]
	\column{0.5\textwidth}
	\begin{itemize}
		\item 按照满足极小化极大方程的关系公式,画出一条$P_D-P_F$直线, 该直线与给定信噪比下的$P_D-P_F$工作特性曲线相交。如, $d=d_1$, 交点$b$
		\item 交点即是在极小化极大准则条件下的两种判决概率。
	\end{itemize}
	\column{0.5\textwidth}
	\includegraphics[scale=0.5]{PD-PF-e}
\end{columns}
\textbf{\textcolor{blue}{满足极小化极大方程}}
\begin{align*}
(c_{01}-c_{11})P_D(P_{1g}^\ast)+(c_{10}-c_{00})P_F(P_{1g}^\ast)-c_{01}+c_{00}=0
\end{align*}
\end{frame}

\begin{frame}[shrink]{奈曼---皮尔逊准则}
\begin{columns}[T]
	\column{0.5\textwidth}
	\begin{itemize}
		\item 由$P_F=\alpha$画一条直线
		\item 该直线与给定信噪比下的$P_D-P_F$工作特性曲线相交。如, $d=d_1$, 交点$c$
		\item 交点即是在奈曼---皮尔逊准则下的两种判决概率。
	\end{itemize}
	\column{0.5\textwidth}
	\includegraphics[scale=0.5]{PD-PF-e}
\end{columns}
\end{frame}