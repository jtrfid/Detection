%%%%%%%%%%%%%%%%%%%%%%%%%%ch4-3
\begin{frame}[shrink]
  \frametitle{ch4.信号波形的检测}
  \framesubtitle{ch4-4. 一般二元信号波形的检测---充分统计量的方法}
  \tableofcontents[hideallsubsections]
\end{frame}

\section{一般二元信号波形的检测---充分统计量的方法}

\begin{frame}{一般二元信号波形的检测---充分统计量的方法}
\begin{itemize}
	\setlength{\itemsep}{.5cm}
	\item \textbf{条件: }功率谱密度为$P_n(\omega)=N_0/2$的高斯白噪声背景中一般二元信号波形检测
	\item \textbf{正交级数展开法: }信道噪声是白噪声,正交函数集可任意选取。
	\item \textbf{充分统计量法: }\textcolor{blue}{选取特定的正交函数集},使得有关发送信号的信息只包含在有限的展开系数中。	
\end{itemize}
\end{frame}

\begin{frame}{一般二元信号波形的检测---充分统计量的方法}
信号模型
\begin{align*}
&H_0: x(t)=s_0(t)+n(t), 0\le t\le T\\
&H_1: x(t)=s_1(t)+n(t), 0\le t\le T
\end{align*}
$n(t)$为零均值高斯白噪声
\begin{block}{\textbf{\textcolor{blue}{正交函数集$\{f_k(t)\}$的构造问题}}}
波形相关系数$\rho$:
\[ \rho=\frac{1}{\sqrt{E_{0}E_{1}}}\int_{0}^{T}s_0(t)s_1(t)dt,\quad(|\rho|\le 1) \]
$\rho =0$时, 信号$s_0(t)$与$s_1(t)$正交。\\
$\rho\ne 0$时, 信号$s_0(t)$与$s_1(t)$不正交。	
\end{block}
\end{frame}

\begin{frame}{一般二元信号波形的检测---充分统计量的方法}
$H_0: x(t)=s_0(t)+n(t), 0\le t\le T$\\
$H_1: x(t)=s_1(t)+n(t), 0\le t\le T$\\
\textbf{\textcolor{blue}{(1) 选择一组完备正交函数集,构造两个坐标函数:}}\\
\textbf{第一个坐标函数满足:}\\
\[f_1(t)=\frac{1}{\sqrt{E_1}}s_1(t)\]
$f_1(t)$为确知信号$s_1(t)$的归一化函数, $E_1=\int_{0}^{T}s_1^2(t)dt$\\
其余坐标函数$f_k(t), k\ge 2$是与$f_1(t)$正交, 且两两正交的任意归一化函数, 即
\[f_j(t)\text{和}f_k(t)\text{是正交的}, k\ge 1, j\ge 1, k\ne j \]
\end{frame}

\begin{frame}[shrink]{格拉姆---施密特法构造$f_2(t)$}
\textbf{格拉姆---施密特(Gram---Schmidt)正交化法构造第二个坐标函数: }\\
~\\
利用$s_0(t)$构造与$f_1(t)$正交的信号$g_2(t)$, 使$s_0(t)$在$f_1(t)$上的投影$s_1$为零。\\
\begin{align*}
g_2(t)&=s_0(t)-s_1f_1(t)\\
=&s_0(t)-\left[\int_{0}^{T}s_0(t)f_1(t)dt\right]f_1(t)\\
&=s_0(t)-\left[\int_{0}^{T}s_0(t)\frac{1}{\sqrt{E_1}}s_1(t)dt\right]\frac{1}{\sqrt{E_1}}s_1(t) &&\text{by }f_1(t)=\frac{1}{\sqrt{E_1}}s_1(t)\\
&=s_0(t)-\rho\sqrt{\frac{E_0}{E_1}}s_1(t) &&\text{by }\rho=\frac{1}{\sqrt{E_{0}E_{1}}}\int_{0}^{T}s_0(t)s_1(t)dt\\
\end{align*}
\end{frame}

\begin{frame}[shrink]{格拉姆---施密特法构造$f_2(t)$}
\begin{align*}
&g_2(t)=s_0(t)-\rho\sqrt{\frac{E_0}{E_1}}s_1(t)\\
&\int_{0}^{T}s_0^2(t)dt=E_0, \int_{0}^{T}s_1^2(t)dt=E_1, \int_{0}^{T}s_0(t)s_1(t)dt=\rho\sqrt{E_0E_1}
\end{align*}
归一化$g_2(t)$, 得到\textbf{第二个坐标函数:}
\begin{align*}
f_2(t)&=\frac{g_2(t)}{\sqrt{\int_{0}^{T}g_2^2(t)dt}}=\frac{s_0(t)-\rho\sqrt{\frac{E_0}{E_1}}s_1(t)}{\sqrt{\int_{0}^{T}\left(s_0(t)-\rho\sqrt{\frac{E_0}{E_1}}s_1(t)\right)^2dt}}\\
&=\frac{s_0(t)-\rho\sqrt{\frac{E_0}{E_1}}s_1(t)}{\sqrt{\int_{0}^{T}\left(s_0^2(t)-2\rho\sqrt{\frac{E_0}{E_1}}s_0(t)s_1(t)+\rho^2\frac{E_0}{E_1}s_1^2(t)\right)dt}}\\
&=\frac{s_0(t)-\rho\sqrt{\frac{E_0}{E_1}}s_1(t)}{\sqrt{E_0-2\rho\sqrt{\frac{E_0}{E_1}}\sqrt{E_0E_1}\rho}+\rho^2\frac{E_0}{E_1}E_1}
=\frac{1}{\sqrt{(1-\rho^2)E_0}}\left(s_0(t)-\rho\sqrt{\frac{E_0}{E_1}}s_1(t)\right)
\end{align*}
\end{frame}

\begin{frame}[shrink]{证明$f_1(t)$和$f_2(t)$是正交函数集的前两个坐标函数}
\begin{align*}
&f_1(t)=\frac{1}{\sqrt{E_1}}s_1(t),\quad 0\le t\le T\\
&f_2(t)=\frac{1}{\sqrt{(1-\rho^2)E_0}}\left(s_0(t)-\rho\sqrt{\frac{E_0}{E_1}}s_1(t)\right),\quad 0\le t\le T\\
&\int_{0}^{T}s_0^2(t)dt=E_0, \int_{0}^{T}s_1^2(t)dt=E_1, \int_{0}^{T}s_0(t)s_1(t)dt=\rho\sqrt{E_0E_1}
\end{align*}
证明: $f_1(t)$和$f_2(t)$满足正交集坐标函数的定义。\\
先证明$f_1(t), f_2(t)$是归一化函数。因为
\begin{align*}
\int_{0}^{T}f_1^2(t)dt&=\frac{1}{E_1}\int_{0}^{T}s_1^2(t)dt=1\\
\int_{0}^{T}f_2^2(t)dt&=\frac{1}{(1-\rho^2)E_0}\int_{0}^{T}\left(s_0(t)-\rho\sqrt{\frac{E_0}{E_1}}s_1(t)\right)^2dt\\
&=\frac{1}{(1-\rho^2)E_0}\int_{0}^{T}\left(s_0^2(t)-2\rho\sqrt{\frac{E_0}{E_1}}s_0(t)s_1(t)+\rho^2\frac{E_0}{E_1}s_1^2(t)\right)dt\\
&=\frac{1}{(1-\rho^2)E_0}\left(E_0-2\rho\sqrt{\frac{E_0}{E_1}}\sqrt{E_0E_1}\rho+\rho^2\frac{E_0}{E_1}E_1\right)=1
\end{align*}
\end{frame}

\begin{frame}[shrink]{证明$f_1(t)$和$f_2(t)$是正交函数集的前两个坐标函数}
\begin{align*}
&f_1(t)=\frac{1}{\sqrt{E_1}}s_1(t),\quad 0\le t\le T\\
&f_2(t)=\frac{1}{\sqrt{(1-\rho^2)E_0}}\left(s_0(t)-\rho\sqrt{\frac{E_0}{E_1}}s_1(t)\right),\quad 0\le t\le T\\
&\int_{0}^{T}s_0^2(t)dt=E_0, \int_{0}^{T}s_1^2(t)dt=E_1, \int_{0}^{T}s_0(t)s_1(t)dt=\rho\sqrt{E_0E_1}
\end{align*}
证明: 
再证明$f_1(t), f_2(t)$是相互正交的两个函数。因为
\begin{align*}
\int_{0}^{T}f_1(t)f_2(t)dt&=\int_{0}^{T}\frac{1}{\sqrt{E_1}}s_1(t)
\frac{1}{\sqrt{(1-\rho^2)E_0}}\left(s_0(t)-\rho\sqrt{\frac{E_0}{E_1}}s_1(t)\right)dt\\
&=\frac{1}{\sqrt{(1-\rho)^2E_0E_1}}\left(\int_{0}^{T}s_0(t)s_1(t)dt-\rho\sqrt{\frac{E_0}{E_1}}\int_{0}^{T}s_1^2(t)dt\right)\\
&=\frac{1}{\sqrt{(1-\rho)^2E_0E_1}}\left(\rho\sqrt{E_0E_1}-\rho\sqrt{\frac{E_0}{E_1}}E_1\right)=0
\end{align*}
所以, $f_1(t), f_2(t)$是相互正交的两个函数。\\
综上, $f_1(t), f_2(t)$是归一化函数, 且满足正交性, 是正交函数集的前两个坐标函数。$\hfill\square$
\end{frame}

\begin{frame}{充分统计量的方法, 选择一组完备正交函数集}
\begin{align*}
&f_1(t)=\frac{1}{\sqrt{E_1}}s_1(t),\quad 0\le t\le T\\
&f_2(t)=\frac{1}{\sqrt{(1-\rho^2)E_0}}\left(s_0(t)-\rho\sqrt{\frac{E_0}{E_1}}s_1(t)\right),\quad 0\le t\le T
\end{align*}
~\\
\vspace{0.2cm}
其余坐标函数$f_k(t), k\le 3$是与$f_1(t)$和$f_2(t)$正交, 且两两相互正交的任意归一化函数, 即$f_j(t)$和$f_k(t)$是正交的, $k\ge 1, j\ge 1, k\ne j$
\[\int_{0}^{T}f_j(t)f_k(t)dt=0,\quad k\ge 1, j\ge 1, k\ne j\]
\end{frame}

\begin{frame}[shrink]{对接收信号进行正交展开(假设$H_0: x_1$)}
\begin{block}{假设$H_0:x(t)=s_0(t)+n(t)$下,展开系数$x_1$}
\begin{align*}
x_1&=\int_{0}^{T}x(t)f_1(t)dt=\int_{0}^{T}[s_0(t)+n(t)]f_1(t)dt=\int_{0}^{T}s_0(t)f_1(t)dt+\int_{0}^{T}n(t)f_1(t)dt\\
&=\int_{0}^{T}s_0(t)[\frac{1}{\sqrt{E_1}}s_1(t)]dt+n_1=\frac{1}{\sqrt{E_1}}\int_{0}^{T}s_0(t)s_1(t)dt+n_1\\
&=\rho\sqrt{E_0}+n_1
\end{align*}
\end{block}
\begin{align*}
&f_1(t)=\frac{1}{\sqrt{E_1}}s_1(t),\quad \int_{0}^{T}n(t)f_1(t)dt=n_1\\
&\rho=\frac{1}{\sqrt{E_0E_1}}\int_{0}^{T}s_0(t)s_1(t)dt\implies \int_{0}^{T}s_0(t)s_1(t)dt=\rho\sqrt{E_0E_1}
\end{align*}
\end{frame}

\begin{frame}[shrink]{对接收信号进行正交展开(假设$H_0:x_2$)}
\begin{block}{假设$H_0:x(t)=s_0(t)+n(t)$下,展开系数$x_2$}
	\begin{align*}
	x_2&=\int_{0}^{T}x(t)f_2(t)dt=\int_{0}^{T}[s_0(t)+n(t)]f_2(t)dt=\int_{0}^{T}s_0(t)f_2(t)dt+\int_{0}^{T}n(t)f_2(t)dt\\
	&=\int_{0}^{T}s_0(t)\frac{1}{\sqrt{(1-\rho^2)E_0}}\left(s_0(t)-\rho\sqrt{\frac{E_0}{E_1}}s_1(t)\right)dt+n_2\\
	&=\frac{1}{\sqrt{(1-\rho^2)E_0}}\left[\int_{0}^{T}s_0^2(t)dt-\rho\sqrt{\frac{E_0}{E_1}}\int_{0}^{T}s_0(t)s_1(t)dt\right]+n_2\\
	&=\frac{1}{\sqrt{(1-\rho^2)E_0}}\left[E_0-\rho\sqrt{\frac{E_0}{E_1}}\rho\sqrt{E_0E_1}\right]+n_2=\sqrt{(1-\rho^2)E_0}+n_2
	\end{align*}
\end{block}
\begin{align*}
&f_2(t)=\frac{1}{\sqrt{(1-\rho^2)E_0}}\left(s_0(t)-\rho\sqrt{\frac{E_0}{E_1}}s_1(t)\right),\quad E_0=\int_{0}^{T}s_0^2(t)dt,\quad \int_{0}^{T}n(t)f_2(t)dt=n_2\\
&\rho=\frac{1}{\sqrt{E_0E_1}}\int_{0}^{T}s_0(t)s_1(t)dt\implies \int_{0}^{T}s_0(t)s_1(t)dt=\rho\sqrt{E_0E_1}
\end{align*}
\end{frame}

\begin{frame}[shrink]{对接收信号进行正交展开(假设$H_0: x_k$)}
\begin{block}{假设$H_0:x(t)=s_0(t)+n(t)$下,展开系数$x_k$}
	\begin{align*}
	x_k&=\int_{0}^{T}x(t)f_k(t)dt=\int_{0}^{T}[s_0(t)+n(t)]f_k(t)dt=\int_{0}^{T}s_0(t)f_k(t)dt+\int_{0}^{T}n(t)f_k(t)dt\\
	&=0+\int_{0}^{T}n(t)f_k(t)dt=n_k\quad k\ge 3
	\end{align*}
\end{block}
\begin{align*}
&f_2(t)=\frac{1}{\sqrt{(1-\rho^2)E_0}}\left(s_0(t)-\rho\sqrt{\frac{E_0}{E_1}}s_1(t)\right), s_1(t)=\sqrt{E_1}f_1(t)\\
&\implies s_0(t)=\left(\sqrt{(1-\rho^2)E_0}\right)f_2(t)+\left(\rho\sqrt{E_0}\right)f_1(t)\\
&\int_{0}^{T}f_j(t)f_k(t)dt=0,\quad k\ge 1, j\ge 1, k\ne j
\end{align*}
\end{frame}

\begin{frame}[shrink]{对接收信号进行正交展开(假设$H_1$)}
\begin{block}{假设$H_1:x(t)=s_1(t)+n(t)$下,展开系数}
\begin{align*}
	x_1&=\int_{0}^{T}x(t)f_1(t)dt=\int_{0}^{T}[s_1(t)+n(t)]f_1(t)dt=\int_{0}^{T}s_1(t)f_1(t)dt+\int_{0}^{T}n(t)f_1(t)dt\\
	&=\int_{0}^{T}s_1(t)[\frac{1}{\sqrt{E_1}}s_1(t)]dt+n_1=\frac{1}{\sqrt{E_1}}\int_{0}^{T}s_1^2(t)dt+n_1\\
	&=\sqrt{E_1}+n_1\quad (\text{by }f_1(t)=\frac{1}{\sqrt{E_1}}s_1(t),E_1=\int_{0}^{T}s_1^2(t)dt)\\
	x_2&=\int_{0}^{T}x(t)f_2(t)dt=\int_{0}^{T}[s_1(t)+n(t)]f_2(t)dt=\int_{0}^{T}s_1(t)f_2(t)dt+\int_{0}^{T}n(t)f_2(t)dt\\
	&=\int_{0}^{T}[\sqrt{E_1}f_1(t)]f_2(t)dt+n_2=0+n_2=n_2\\
	x_k&=\int_{0}^{T}x(t)f_k(t)dt=\int_{0}^{T}[s_1(t)+n(t)]f_k(t)dt=\int_{0}^{T}[\sqrt{E_1}f_1(t)+n(t)]f_k(t)dt\\
	&=\int_{0}^{T}n(t)f_k(t)dt=n_k\quad k\ge 3\quad (by\quad s_1(t)=\sqrt{E_1}f_1(t), \int_{0}^{T}f_1(t)f_k(t)dt=0,k\ge 3)
\end{align*}
\end{block}
\end{frame}

\begin{frame}[shrink]{充分量统计法}
\textbf{\textcolor{blue}{(2)利用构造的正交函数集$f_1(t),f_2(t)$和$\{f_k(t)|k\ge 3\}$对接收信号进行正交展开}}
\begin{block}{两个假设下展开系数$x_1,x_2$}
\begin{align*}
&H_0: x_1=\int_{0}^{T}x(t)f_1(t)dt=\int_{0}^{T}[s_0(t)+n(t)]f_1(t)dt=\rho\sqrt{E_0}+n_1\\
&H_0: x_2=\int_{0}^{T}x(t)f_2(t)dt=\int_{0}^{T}[s_0(t)+n(t)]f_2(t)dt=\sqrt{(1-\rho^2)E_0}+n_2\\
&H_1: x_1=\int_{0}^{T}x(t)f_1(t)dt=\int_{0}^{T}[s_1(t)+n(t)]f_1(t)dt=\sqrt{E_1}+n_1\\
&H_1: x_2=\int_{0}^{T}x(t)f_2(t)dt=\int_{0}^{T}[s_1(t)+n(t)]f_2(t)dt=n_2\\
&H_0,H_1: x_k=\int_{0}^{T}x(t)f_k(t)dt=n_k \quad (k\ge 3)\implies\text{不含确知信号$s_0(t),s_1(t)$信息}
\end{align*}
\end{block}
\textbf{\textcolor{blue}{$\bm{x}=(x_1,x_2)^T$是充分统计量。且$x_1$和$x_2$为高斯随机变量, 相互统计独立。}}
\end{frame}

\begin{frame}[shrink]{充分量统计法: $x_1,x_2$的均值和方差}
\begin{align*}
&E[x_1|H_0]=E\left[\rho\sqrt{E_0}+n_1\right]=\rho\sqrt{E_0}\\
&E[x_2|H_0]=E\left[\sqrt{(1-\rho^2)E_0}+n_2\right]=\sqrt{(1-\rho^2)E_0}\\
&Var[x_1|H_0]=Var[x_2|H_0]=E[n_1^2]=E[n_2^2]=\frac{N_0}{2}\\
&E[x_1|H_1]=E\left[\sqrt{E_1}+n_1\right]=\sqrt{E_1}\\
&E[x_2|H_1]=E[n_2]=0\\
&Var[x_1|H_1]=Var[x_2|H_1]=E[n_1^2]=E[n_2^2]=\frac{N_0}{2}\\
\end{align*}
\end{frame}

\begin{frame}[shrink]{充分量统计法---构建似然比}
\textbf{\textcolor{blue}{(3)利用得到的展开系数,构建似然比表达式
}}
\begin{align*}
\bm{x}=(x_1,x_2)^T\\
\lambda(\bm{x})=\frac{p(\bm{x}|H_1)}{p(\bm{x}|H_0)}\mathop{\gtrless}_{H_0}^{H_1}\eta\\
\lambda(\bm{x})=\frac{p(x_1,x_2|H_1)}{p(x_1,x_2|H_0)}\mathop{\gtrless}_{H_0}^{H_1}\eta\\
\frac{\frac{1}{\sqrt{\pi N_0}}\exp\left(-\frac{\left(x_1-\sqrt{E_1}\right)^2}{N_0}\right)\frac{1}{\sqrt{\pi N_0}}\exp\left(-\frac{x_2^2}{N_0}\right)}
{\frac{1}{\sqrt{\pi N_0}}\exp\left(-\frac{\left(x_1-\rho\sqrt{E_0}\right)^2}{N_0}\right)\frac{1}{\sqrt{\pi N_0}}\exp\left(-\frac{\left(x_2-\sqrt{(1-\rho^2)E_0}\right)^2}{N_0}\right)}
\mathop{\gtrless}_{H_0}^{H_1}\eta
\end{align*}
\end{frame}

%%%%%%%%%%%%%%%%%%%%%%%%%%%%%%%%%%%%%%%%%%%%%%%%%%%%%%%%%
\begin{frame}[shrink]{充分量统计法---充分统计量$x_1$}
\textbf{\textcolor{blue}{(3)利用得到的展开系数,构建似然比表达式
}}\\
展开系数中只有$x_1$含有接收信号的信息,因此展开系数$x_1$是一个\textbf{充分统计量}。利用$x_1$构成的似然比检验,表示为
\[\lambda(x_1)=\frac{p(x_1|H_1)}{p(x_1|H_0)}\mathop{\gtrless}_{H_0}^{H_1}\eta\]
因为
\begin{align*}
&f_1(t)=\frac{1}{\sqrt{E_s}}s(t),\quad (x_1|H_0)=n_1,\quad (x_1|H_1)=\sqrt{E_s}+n_1\\ &x_1=\int_{0}^{T}x(t)f_1(t)dt=\frac{1}{\sqrt{E_s}}\int_{0}^{T}x(t)s(t)dt
\end{align*}
~\\
因为充分统计量$x_1$是高斯随机变量,可用假设$H_0$和假设$H_1$下的均值和方差表示。
\end{frame}

\begin{frame}[shrink]{充分量统计法---充分统计量$x_1$}
\textbf{\textcolor{blue}{(3)利用得到的展开系数,构建似然比表达式
}}\\
展开系数中只有$x_1$含有接收信号的信息,因此展开系数$x_1$是一个充分统计量。利用$x_1$构成的似然比检验,表示为
\[\lambda(x_1)=\frac{p(x_1|H_1)}{p(x_1|H_0)}\mathop{\gtrless}_{H_0}^{H_1}\eta\]
因为
\begin{align*}
&f_1(t)=\frac{1}{\sqrt{E_s}}s(t),\quad (x_1|H_0)=n_1,\quad (x_1|H_1)=\sqrt{E_s}+n_1\\ &x_1=\int_{0}^{T}x(t)f_1(t)dt=\frac{1}{\sqrt{E_s}}\int_{0}^{T}x(t)s(t)dt
\end{align*}
~\\
所以充分统计量$x_1$是高斯随机变量,可用假设$H_0$和假设$H_1$下的均值和方差表示。
\end{frame}

\begin{frame}{充分量统计法---充分统计量$x_1$}
$x_1$是高斯变量, 均值和方差
\begin{align*}
&E[x_1|H_0]=0,&Var[x_1|H_0]=\frac{N_0}{2}\\
&E[x_1|H_1]=\sqrt{E_s},&Var[x_1|H_1]=\frac{N_0}{2}
\end{align*}
概率密度函数
\begin{align*}
p(x_1|H_0)&=\left(\frac{1}{\pi N_0}\right)^{1/2}\exp\left(-\frac{x_1^2}{N_0}\right)\\
p(x_1|H_1)&=\left(\frac{1}{\pi N_0}\right)^{1/2}\exp\left(-\frac{(x_1-\sqrt{E_s})^2}{N_0}\right)
\end{align*}
\end{frame}

\begin{frame}[shrink]{推导$E[x_1|H_0]$}
\begin{align*}
&f_1(t)=\frac{1}{\sqrt{E_s}}s(t),\quad (x_1|H_0)=n_1,\quad (x_1|H_1)=\sqrt{E_s}+n_1\\ &x_1=\int_{0}^{T}x(t)f_1(t)dt=\frac{1}{\sqrt{E_s}}\int_{0}^{T}x(t)s(t)dt\\
&E[x_1|H_0]=E[n_1]=0
\end{align*}
或:
\begin{align*}
E[x_1|H_0]&=E\left[\frac{1}{\sqrt{E_s}}\int_{0}^{T}x(t)s(t)dt\right] &\text{by }H_0: x(t)=n(t)\\
&=E\left[\frac{1}{\sqrt{E_s}}\int_{0}^{T}n(t)s(t)dt\right]&\\
&=\frac{1}{\sqrt{E_s}}\int_{0}^{T}E[n(t)]s(t)dt=0 &\text{by }E[n(t)]=0
\end{align*}
\end{frame}

\begin{frame}[shrink]{推导$Vax[x_1|H_0]$(方法1)}
\begin{align*}
&H_0:x(t)=n(t),\quad E(x_1|H_0)=0,\quad E_s=\int_{0}^{T}s^2(t)dt\\
&E[n(t)n(u)]=r_n(t-u)=\frac{N_0}{2}\delta(t-u)=\frac{N_0}{2},(t=u,\delta(t-u)=1)\\
&f_1(t)=\frac{1}{\sqrt{E_s}}s(t),\quad (x_1|H_0)=n_1,\quad (x_1|H_1)=\sqrt{E_s}+n_1\\ &x_1=\int_{0}^{T}x(t)f_1(t)dt=\frac{1}{\sqrt{E_s}}\int_{0}^{T}x(t)s(t)dt,\quad (x_1|H_0)=\frac{1}{\sqrt{E_s}}\int_{0}^{T}n(t)s(t)dt
\end{align*}
\begin{align*}
&Var[x_1|H_0]=E[((x_1|H_0)-E(x_1|H_0))^2]=E[(x_1|H_0)^2]=E\left[\left(\frac{1}{\sqrt{E_s}}\int_{0}^{T}n(t)s(t)dt\right)^2\right]\\
&=\frac{1}{E_s}E\left[\int_{0}^{T}n(t)s(t)dt\int_{0}^{T}n(t)s(t)dt\right]=\frac{1}{E_s}E\left[\int_{0}^{T}n(t)s(t)dt\int_{0}^{T}n(u)s(u)du\right]\\
&=\frac{1}{E_s}\int_{0}^{T}s(t)\left\{\int_{0}^{T}E[n(u)n(t)]s(u)du\right\}dt=\frac{1}{E_s}\int_{0}^{T}s(t)\left[\int_{0}^{T}\frac{N_0}{2}\delta(t-u)s(u)du\right]dt\\
&=\frac{N_0}{2E_s}\int_{0}^{T}s(t)\left(\int_{0}^{T}s(u)du\right)dt=\frac{N_0}{2E_s}\int_{0}^{T}s^2(t)dt=\frac{N_0}{2}
\end{align*}
\end{frame}

\begin{frame}[shrink]{推导$Vax[x_1|H_0]$(方法2)}
\begin{align*}
&H_0: x(t)=n(t),\quad E(x_1|H_0)=0,\quad E_s=\int_{0}^{T}s^2(t)dt\\
&E[n(t)n(u)]=r_n(t-u)=\frac{N_0}{2}\delta(t-u)=\frac{N_0}{2},(t=u,\delta(t-u)=1)\\
&f_1(t)=\frac{1}{\sqrt{E_s}}s(t),\quad (x_1|H_0)=n_1,\quad (x_1|H_1)=\sqrt{E_s}+n_1\\ &x_1=\int_{0}^{T}x(t)f_1(t)dt=\frac{1}{\sqrt{E_s}}\int_{0}^{T}x(t)s(t)dt,\quad (x_1|H_0)=\frac{1}{\sqrt{E_s}}\int_{0}^{T}n(t)s(t)dt\\
&n_1=\int_{0}^{T}n(t)f_1(t)dt=\int_{0}^{T}n(t)\frac{1}{\sqrt{E_s}}s(t)dt=\frac{1}{\sqrt{E_s}}\int_{0}^{T}n(t)s(t)dt
\end{align*}
\begin{align*}
Var[x_1|H_0]&=E[((x_1|H_0)-E(x_1|H_0))^2]=E[(x_1|H_0)^2]=E[n_1^2]\\
&=E\left[\left(\frac{1}{\sqrt{E_s}}\int_{0}^{T}n(t)s(t)dt\right)^2\right]=\frac{N_0}{2}\qquad (\text{同方法1})
\end{align*}
\end{frame}

\begin{frame}{推导$p(x_1|H_0)$}
\[ E[x_1|H_0]=0,\quad Var[x_1|H_0]=\frac{N_0}{2}\]
\begin{align*}
p(x_1|H_0)&=\left(\frac{1}{2\pi Var[x_1|H_0]}\right)^{1/2}\exp\left(-\frac{(x_1-E[x_1|H_0])^2}{2Var[x_1|H_0]}\right)\\
&=\left(\frac{1}{\pi N_0}\right)^{1/2}\exp\left(-\frac{x_1^2}{N_0}\right)
\end{align*}
\end{frame}

\begin{frame}{推导$E[x_1|H_1]$和$Var[x_1|H_1]$}
\begin{align*}
&f_1(t)=\frac{1}{\sqrt{E_s}}s(t),\quad E_s=\int_{0}^{T}s^2(t)dt, \quad n_1=\int_{0}^{T}n(t)f_1(t)dt\\
&(x_1|H_1)=\int_{0}^{T}x(t)f_1(t)dt=\int_{0}^{T}[s(t)+n(t)]f_1(t)dt=\sqrt{E_s}+n_1
\end{align*}
\begin{align*}
E[x|H_1]&=E\left[\sqrt{E_s}+n_1\right]\\
&=E[\sqrt{E_s}]+E[n_1]\\
&=E[\sqrt{E_s}]=\sqrt{E_s}\\
Var[x_1|H_1]&=E[((x|H_1)-E(x|H_1))^2]=E[(\sqrt{E_s}+n_1-\sqrt{E_s})]\\
&=E[n_1^2]=Var[x_1|H_0]\\
&=\frac{N_0}{2}
\end{align*}
\end{frame}

\begin{frame}{推导$p(x_1|H_1)$}
\[ E[x_1|H_1]=\sqrt{E_s},\quad Var[x_1|H_1]=\frac{N_0}{2}\]
\begin{align*}
p(x_1|H_1)&=\left(\frac{1}{2\pi Var[x_1|H_1]}\right)^{1/2}\exp\left(-\frac{(x_1-E[x_1|H_1])^2}{2Var[x_1|H_1]}\right)\\
&=\left(\frac{1}{\pi N_0}\right)^{1/2}\exp\left(-\frac{(x_1-\sqrt{E_s})^2}{N_0}\right)
\end{align*}
\end{frame}

\begin{frame}{充分量统计法---充分统计量$x_1$}
$x_1$是高斯变量, 均值和方差
\begin{align*}
&E[x_1|H_0]=0,&Var[x_1|H_0]=\frac{N_0}{2}\\
&E[x_1|H_1]=\sqrt{E_s},&Var[x_1|H_1]=\frac{N_0}{2}
\end{align*}
概率密度函数
\begin{align*}
p(x_1|H_0)&=\left(\frac{1}{\pi N_0}\right)^{1/2}\exp\left(-\frac{x_1^2}{N_0}\right)\\
p(x_1|H_1)&=\left(\frac{1}{\pi N_0}\right)^{1/2}\exp\left(-\frac{(x_1-\sqrt{E_s})^2}{N_0}\right)
\end{align*}
\end{frame}

\begin{frame}[shrink]{充分量统计法---判决表达式}
\begin{align*}
\lambda(x_1)=\frac{p(x_1|H_1)}{p(x_1|H_0)}&\mathop{\gtrless}_{H_0}^{H_1}\eta\\
\frac{\left(\frac{1}{\pi N_0}\right)^{1/2}\exp\left(-\frac{(x_1-\sqrt{E_s})^2}{N_0}\right)}{\left(\frac{1}{\pi N_0}\right)^{1/2}\exp\left(-\frac{x_1^2}{N_0}\right)}&\mathop{\gtrless}_{H_0}^{H_1}\eta\implies x_1\mathop{\gtrless}_{H_0}^{H_1}\frac{N_0}{2\sqrt{E_s}}\ln\eta+\frac{\sqrt{E_s}}{2}
\end{align*}
\[\text{代入展开系数 }\quad x_1=\int_{0}^{T}x(t)f_1(t)dt=\frac{1}{\sqrt{E_s}}\int_{0}^{T}x(t)s(t)dt \]
\textbf{化简得, 判决表达式:}
\[l[x(t)]\mathop{=}^{def}\int_{0}^{T}x(t)s(t)dt\mathop{\gtrless}_{H_0}^{H_1}\frac{N_0}{2}\ln\eta+\frac{E_s}{2}\mathop{=}^{def}\gamma \]
\begin{block}{结论}
	由任意正交函数集对$x(t)$进行正交级数展开法与由充分统计量法导出的判决表达式是完全一样的,因而也具有相同的检测系统结构和相同的检测性能。
\end{block}
\end{frame}

\section{简单二元信号波形的检测---总结}

\begin{frame}{简单二元信号波形的检测---总结}
\begin{enumerate}
	\setlength{\itemsep}{.5cm}
	\item 首先,利用随机过程的正交级数展开,将随机过程用一组随机变量来表示;
	\item 然后,针对展开得到的随机变量,利用第三章的统计检测方法,构建贝叶斯检测表达式;
	\item 最后,利用展开系数与随机过程之间的表示关系,构建波形信号的检测表达式。
	\item \textbf{两种方法:正交级数展开和充分统计量, 所得结果相同。}
\end{enumerate}
\end{frame}
