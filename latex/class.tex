% !Mode::"TeX:UTF-8"
% !Mode:: "TeX:UTF-8"
\documentclass[xcolor=svgnames,serif,table,10pt]{beamer}
\mode<presentation>{
% Setup appearance:
\useoutertheme{infolines}
\usetheme{Darmstadt}
\setbeamercovered{transparent}
\setbeamertemplate{caption}[numbered]
\setbeamertemplate{navigation symbols}{}
\setbeamertemplate{blocks}[rounded][shadow=true]
\setbeamertemplate{enumerate items}[circle]

% 修改样式
\setbeamercolor{box}{bg=black!20!orange,fg=white}
\setbeamercolor{block title}{use=sidebar,fg=sidebar.fg!10!white,bg=orange!70!black}
\setbeamercolor{block title example}{use=sidebar,fg=sidebar.fg!10!white,bg=black!60!green}
\setbeamercolor{block title alerted}{use=sidebar,fg=sidebar.fg!10!white,bg=black!50!red}

\setbeamertemplate{headline}
{%
  \begin{beamercolorbox}[shadow=true]{section in head/foot}
  \vskip2pt\insertnavigation{\paperwidth}\vskip2pt
  \end{beamercolorbox}%
}
}
\usepackage{url}
\usepackage{animate}
\usepackage[english]{babel}
\usepackage{times}
\usepackage[T1]{fontenc}
\usepackage{multirow,multicol,longtable}
\usepackage{graphics}
\usepackage{xcolor}
\usepackage[no-math]{fontspec}%-------------------------------------------------- 提供字体选择命令
\usepackage{xunicode}%----------------------------------------------------------- 提供Unicode字符宏
\usepackage{xltxtra}%------------------------------------------------------------ 提供了针对XeTeX的改进并且加入了XeTeX的LOGO
\usepackage[BoldFont,SlantFont,CJKchecksingle]{xeCJK}%--------------------------- 使用xeCJK宏包


%================================== 设置中文字体 ================================%
%\setCJKmainfont{Adobe Heiti Std}%------------------------------------------------设置正文为黑体
%\setCJKmonofont{Adobe Song Std}%-------------------------------------------------设置等距字体
%\setCJKsansfont{Adobe Kaiti Std}%------------------------------------------------设置无衬线字体
% \setCJKfamilyfont{zxzt}{FZShouJinShu-S10S}
% \setCJKfamilyfont{FZDH}{FZDaHei-B02S}
%================================== 设置中文字体 ================================%

%================================== 设置英文字体 ================================%
\setmainfont[Mapping=tex-text]{Times New Roman}%--------------------------------英文衬线字体
\setsansfont[Mapping=tex-text]{Arial}%------------------------------------英文无衬线字体
\setmonofont[Mapping=tex-text]{Courier New}%-------------------------------------英文等宽字体
\newfontfamily\Arial{Arial}
%================================== 设置英文字体 ================================%

%================================== 设置数学字体 ================================%
%\setmathsfont(Digits,Latin,Greek)[Numbers={Lining,Proportional}]{Minion Pro}
%================================== 设置数学字体 ================================%
\punctstyle{kaiming}%------------------------------------------------------------ 开明式标点格式
\usepackage{graphicx}
\usepackage{tikz}
\usetikzlibrary{positioning,backgrounds}
\usetikzlibrary{fadings}
\usetikzlibrary{patterns}
\usetikzlibrary{calc}
\usetikzlibrary{shadings}
\pgfdeclarelayer{background}
\pgfdeclarelayer{foreground}
\pgfsetlayers{background,main,foreground}
\usepackage{xifthen}
\usepackage{colortbl,dcolumn}
\usepackage{enumerate}
\usepackage{pifont}
\usepackage{tabularx}
\usepackage{booktabs}
\usepackage{hyperref}
%=================================== 数学符号 =================================%
\newcommand{\rtn}{\mathrm{\mathbf{R}}}
\newcommand{\N}{\mathrm{\mathbf{N}}}
\newcommand{\As}{\mathrm{a.s.}}
\newcommand{\Ae}{\mathrm{a.e.}}
\newcommand*{\PR}{\mathrm{\mathbf{P}}}
\newcommand*{\EX}{\mathrm{\mathbf{E}}}
\newcommand{\EXlr}[1]{\mathrm{\mathbf{E}}\left[#1\right]}
\newcommand*{\dif}{\,\mathrm{d}}
\newcommand*{\F}{\mathcal{F}}
\newcommand*{\h}{\mathcal{H}}
\newcommand*{\vp}{\varepsilon}
\newcommand*{\prs}{\dif\PR-\As}
\newcommand*{\dte}{\dif t-\Ae}
\newcommand*{\pts}{\dif\PR\times\dif t-\Ae}
\newcommand{\Ito}{It\^{o}}
\newcommand{\tT}[1][0]{[#1,T]}
\newcommand{\intT}[2][T]{\int^{#1}_{#2}}
\newcommand{\intTe}[1][t]{\intT[t+\varepsilon]{#1}}
\newcommand{\s}{\mathcal{S}}
\newcommand{\me}{\mathrm{e}}
\newcommand{\one}[1]{{\bf 1}_{#1}}
\renewcommand{\M}{{\rm M}}
\newcommand{\Me}[1][t]{M^{\varepsilon}_{#1}}
\newcommand{\Ne}[1][t]{N^{\varepsilon}_{#1}}
\newcommand{\Pe}[1][t]{P^{\varepsilon}_{#1}}
\DeclareMathOperator*{\sgn}{sgn}
% =================================== 数学符号 =================================%

% 定义罗马数字
\makeatletter
\newcommand{\rmnum}[1]{\romannumeral #1}
\newcommand{\Rmnum}[1]{\expandafter\@slowromancap\romannumeral #1@}
\makeatother

% 定义破折号
\newcommand{\pozhehao}{\kern0.3ex\rule[0.8ex]{2em}{0.1ex}\kern0.3ex}
% 中文日期
\def\CJK@today{\the\year 年 \the\month 月}
\newcommand\zhtoday{\CJK@today}

% 中文图表
\renewcommand\figurename{图}
\renewcommand\tablename{表}

\graphicspath{{./}}

% Author, Title, etc.

\title{一个简单的汇报}

%% \subtitle{Foreground-constrained Eulerian Video Motion Magnification}

\author[陆嵩]{陆嵩
\\中国科学院数学与系统科学研究院
  \\计算数学与科学工程计算研究所
  \\科学工程计算国家重点实验室}

\institute[LSEC,AMSS,CAS]{\includegraphics[height=1cm]{xd.jpg}}

\date{\zhtoday}

\setlength{\baselineskip}{22pt}
\renewcommand{\baselinestretch}{1.4}

% The main document

\begin{document}
\setlength{\abovedisplayskip}{1ex}%------------------------------------------ 公式前的距离
\setlength{\belowdisplayskip}{1ex}%------------------------------------------ 公式后的距离

\begin{frame}
  \titlepage
\end{frame}

\begin{frame}
  \frametitle{主要内容}
  \tableofcontents[hideallsubsections]
\end{frame}

\section{随机变量}

\begin{frame}
概率论中的三个组成部分:
\begin{itemize}
	\item 样本空间$\Omega$
	\item 事件域$\mathcal{F}$
	\item 概率$P$
\end{itemize}
\end{frame}


\begin{frame}
\begin{itemize}
	\item 样本空间$\Omega$:一个随机试验所有可能出现的结果的全体,称为随机事件的样本空间。
	\item 样本点$\xi_k$:试验的一个结果。$\Omega=\{\xi_k\}$
	\item 随机事件$A$: 样本空间中的某个子集称为随机事件,简称事件(事件是集合)。
    \item 事件域$\mathcal{F}$: 样本空间中的某些子集构成的满足如下条件的集合,称为事件域(又称$\sigma^-$域)。
	\begin{itemize}
		\item[(1)] $\Omega\in\mathcal{F}$
		\item[(2)] 若$A\in\mathcal{F}$, 则$A$的补$\overline{A}\in\mathcal{F}$
		\item[(3)] 若$A_n\in\mathcal{F}$, 则$\bigcap_{n=1}^{\infty}\in\mathcal{F}$
	\end{itemize}
\end{itemize}
\end{frame}

\begin{frame}
  事件域中的元素就是随机事件。如果这些事件的随机性能够由定义在$\mathcal{F}$上的具有非负性,归一性和可列加性的实函数$P(A)$来确定,则称$P$是定义在二元组$(\Omega,\mathcal{F})$上的概率,而称$P(A)$为事件$A$的概率。
  \begin{enumerate}
  	\item[(1)] 非负性。 $P(A)\ge 0$
  	\item[(2)] 归一性。 $P(\mathcal{F})=1$
    \item[(3)] 可列加性。$A_1,A_2,...,A_n$互不相容,则$P(A_1\cup A_2\cup\dots\cup A_n) = P(A_1)+P(A_2)+\cdots+P(A_n)$
  \end{enumerate}
\end{frame}

\begin{frame}
\begin{definition}
	设$(\Omega,\mathcal{F},P)$是一概率空间,$x(\xi)|\xi\in\Omega$是定义在$\Omega$上的单值实函数,如果对任一实数$x$, 集合$\{x(\xi)\le x\}\in\mathcal{F}$, 则称$x(\xi)$为$(\Omega,\mathcal{F},P)$上的一个\textbf{随机变量}。
	
	随机变量$x(\xi)$的定义域为样本空间$\Omega$,它的值域是实数R。所有随机变量$x(\xi)$实际上是一个映射,这个映射为每个来自概率空间的结果$\xi$赋予一个实数$x$。这种映射必须满足条件:
	\begin{itemize}
		\item[(1)] 对任一$x$,集合$\{x(\xi)\le x\}$是这个概率空间中的一个事件,并有确定的概率$P\{x(\xi)\le x\}$;
		\item[(2)] $P\{x(\xi)=\infty \}=0$, $P\{x(\xi)=-\infty \}=0$
	\end{itemize}
\end{definition}
\end{frame}

\begin{frame}
\begin{example}
	抛硬币试验中,H表示正面,T表示反面,样本空间$\Omega=\{H,T\}$,H与T不是数量,不便于计算及理论的研究,因而引入以下变量$\xi$,
	$$x=x(\xi)=
	\begin{cases}
	0, &\xi=T\\
	1, &\xi=H
	\end{cases}
	$$
\end{example}  
\end{frame}

\begin{frame}
\begin{definition}
	设随机试验E的样本空间是$\Omega=\{\xi\}$, 若对于每一个$\xi\Omega$,有一个实数$x(\xi)$与之对应,即$x(\xi)$是定义在$\Omega$上的单值函数,称为随机变量。
\end{definition}
\begin{figure}[htbp]
	\includegraphics[scale=0.4]{xi_map}
\end{figure}
\begin{itemize}
	\item 可用随机变量$x(\xi)$描述事件。\\
	例掷一颗骰子(色子),设出现的点数记为随机事件A,表示``掷出的点数大于3''的事件A,可表示为``$x(\xi)>3$''。反过来,A的一个变化范围表示一个随机事件:``$2<x(\xi)<5$''表示事件``掷出的点数大于2且小于5''。
	\item 随机变量随着试验的结果而取不同的值,在试验之前不能确切知道它取什么值,但是随机变量的取值有一定的统计规律性---概率分布。
\end{itemize}

\end{frame}

\begin{frame}
	关于随机变量(及向量)的研究,是概率论的中心内容.这是因为,对于一个随机试验,我们所关心的往往是与所研究的特定问题有关的某个或某些量,而这些量就是随机变量.
	
	也可以说:\textbf{随机事件}是从静态的观点来研究随机现象,而\textbf{随机变量}则是一种动态的观点,一如数学分析中的常量与变量的区分那样.变量概念是高等数学有别于初等数学的基础概念。同样,概率论能从计算一些孤立事件的概念发展为一个更高的理论体系,其基础概念是随机变量。
\end{frame}


\section{第二部分}
\begin{frame}
  一些内容
\end{frame}
\begin{frame}
  一些内容
\end{frame}
\section{第三部分}
\begin{frame}
  一些内容
\end{frame}
\begin{frame}
  一些内容
\end{frame}




\begin{frame}[plain]{}
  \begin{center}
    \begin{tikzpicture}
      \node[above,xscale=1.2,yscale=1.2]{\Huge 欢迎批评指正!};
    \end{tikzpicture}
  \end{center}
\end{frame}

\end{document}






%%%%下面的内容不参与文档的编译。使用者在想用某个东西时直接可通过查阅,并复制黏贴和修改使用。

\iffalse  %注释开始

%垂直居中
\begin{frame}
  \begin{center}
  需要居中的内容!
  \end{center}
\end{frame}
或者
\begin{frame}
  \centering
  一些内容
\end{frame}

%幻灯片标题的使用
\begin{frame}
\frametitle{第一部分第一张幻灯}
  一些内容
\end{frame}

%项目编号的使用
\begin{frame}
  \frametitle{条目}
  \begin{itemize}
  \item 项目1
  \item 项目2
  \item 项目3
  \item 项目4
    \begin{itemize}
    \item 二级项目1
    \item 二级项目2
    \end{itemize}
  \end{itemize}
\end{frame}

%表格的使用
\begin{frame}
  \frametitle{表格}
  \begin{table}[htbp!]
    \centering
    \caption{主流机器学习框架}
    \begin{tabular}{c|c|c|c|c}
      \toprule[1pt]
      机器学习库	& 机构 & 支持语言  & 平台 & Tensor \\
      \toprule[1pt]
      TensorFlow	& Google & C++,Python &跨平台 & Good \\
 	  \hline
      Pytorch	&  Facebook& Python & 跨平台 & Good \\
 	  \bottomrule[1pt]
    \end{tabular}
  \end{table}
\end{frame}

%区块的使用
\begin{frame}
  \frametitle{分析}
  \begin{block}{XXX 算法}
	\begin{itemize}
		\item 步骤1
	 	\item 步骤2
	 	\item 步骤3
	 \end{itemize}
  \end{block}
\end{frame}

%使用区块来强调内容
\begin{frame}
  \frametitle{强调}
  \begin{itemize}
  \item 这是内容
  \end{itemize}
  \only<1>\begin{block}{}
    这里蹦出来一个强调!
  \end{block}
\end{frame}

%section中目录的使用
\begin{frame}
  \frametitle{技术影响力}
    \tableofcontents[currentsection,hideallsubsections]
\end{frame}

%插入图片
\begin{frame}
\begin{figure}[!h]
  \centering
  % Requires \usepackage{graphicx}
  \includegraphics[width=2cm]{pics/logo.jpg}\\
  \caption{logo图片样例}\label{pic6}
\end{figure}
\end{frame}

%分栏实现图文混排
\begin{frame}
分栏前面的一些内容!!
\begin{columns}%0.6 0.4表示相对比例
\column{0.6\textwidth}%<1->
分栏的左侧,文字叙述。
\column{0.4\textwidth}%<1->
分栏的右侧插入了图片。
 \begin{figure}[!h]
  \centering
  % Requires \usepackage{graphicx}
  \includegraphics[width=4cm]{pics/logo.jpg}\\
  \caption{logo图片样例}\label{pic6}
\end{figure}
\end{columns}
分栏后面的一些内容!!
\end{frame}

\fi   %注释结束
